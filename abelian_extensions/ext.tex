\newpage
\section{Abelian extensions of $\mathbb{Q}(i)$}
\label{sec:abel-extens-Qi}

Let $k = \mathbb{Q}(i)$, so that $\mathfrak{o}_{k} = \mathbb{Z}[i]$.  In Section
\ref{sec:struct-ende-char-zero} we established that the elliptic curve $E$ given by
the equation
\begin{equation*}
  y^{2} = x^{3} + x
\end{equation*}
has complex multiplication by $\mathbb{Z}[i]$.  In particular, we know that $h_{k} =
1$ so $j(E)$ is rational (indeed, we have $j(E) = 1728$).

Consider the group $E[2]$ of $2$-torsion points of $E$.  Recall that a point $P =
(x,y)$ on $E$ has order $2$ if and only if $y = 0$.  So the $x$-co-ordinates are
given by the roots of the equation
\begin{equation*}
  x^{3} + x = 0,
\end{equation*}
which are clearly equal to $0$, $i$ and $-i$.  In particular, the field $K_{2} =
\mathbb{Q}(i)(E[2])$ is generated by elements contained in $\mathbb{Q}(i)$, so that
$K_{2} = \mathbb{Q}(i)$.

Now consider the group of $3$-torsion points of $E$.  There are $8$ non-trivial such
points (since $ \# E[3] = 9$), whose $x$-co-ordinates are determined by the roots of
the quartic equation
\begin{equation}
  \label{eq:3-torsion-polynomial}
  3x^{4} + 6x^{2} - 1.
\end{equation}
Solving the corresponding quadratic in $u = x^{2}$ we see that the roots of
\eqref{eq:3-torsion-polynomial} are $\pm \alpha, \pm \frac{i}{\sqrt{3}\alpha}$, where
\begin{equation*}
  \alpha = \sqrt{ \frac{2\sqrt{3} - 3}{3} }.
\end{equation*}
To find the $y$-co-ordinates, note that
\begin{equation*}
  \alpha^{3} + \alpha = \alpha(\alpha^{2} + 1) = \frac{2\alpha}{\sqrt{3}},
\end{equation*}
from which we condlude
\begin{equation*}
  E[3] = \{ \mathcal{O}, (\alpha,\pm \beta), (-\alpha, \pm \beta),
  (\frac{i}{\sqrt{3}\alpha}, \pm \frac{2 \sqrt{-i}}{\sqrt[4]{27}\beta} ),
  (-\frac{i}{\sqrt{3}\alpha}, \frac{2 \sqrt{i}}{\sqrt[4]{27}\beta} )  \},
\end{equation*}
where
\begin{equation*}
  \beta = \sqrt{ \frac{2\alpha}{\sqrt{3}} }.
\end{equation*}
We claim that $K_{3} = \mathbb{Q}(i)(E[3]) = \mathbb{Q}(i,\beta)$, i.e. the
co-ordinates of the points in $E[3]$ can be described by rational functions of $i$
and $\beta$. Indeed, we have
\begin{align}
  \beta^{2}&= \frac{2\alpha}{\sqrt{3}} \label{eq:beta-powers-1}\\
  \beta^{4}&= \frac{4}{9}(2\sqrt{3} - 3) \label{eq:beta-powers-2}
\end{align}
so \eqref{eq:beta-powers-2} shows that $\sqrt{3}$ belongs to $\mathbb{Q}(i,\beta)$,
which in turn (along with \eqref{eq:beta-powers-1}) shows that $\alpha$ is also in
$\mathbb{Q}(i,\beta)$, as required.

Now, the minimal polynomial for $\beta$ over $\mathbb{Q}(i)$ is
\begin{equation}
  \label{eq:min-poly-beta}
  x^{8} + \frac{8}{3}x^{4} - \frac{16}{27},
\end{equation}
so that $[K_{3} : \mathbb{Q}(i) ] = 8$.  Since \eqref{eq:min-poly-beta} has such a
simple form, we are reduced to solving a quadratic equation (as we did with
\eqref{eq:3-torsion-polynomial}), and we find that the roots of
\eqref{eq:min-poly-beta} are
\begin{equation*}
  \pm \beta, \pm i \beta, \pm \beta^{\prime}, \pm i \beta^{\prime}
\end{equation*}
where
\begin{equation*}
  \beta^{\prime} = 
\end{equation*}


%%% Local Variables: 
%%% mode: latex
%%% TeX-master: "../main"
%%% End: 
