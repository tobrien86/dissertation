\newpage
\section{Abelian extensions of $\mathbb{Q}(i)$}
\label{sec:abel-extens-Qi}

Let $k = \mathbb{Q}(i)$, so that $\mathfrak{o}_{k} = \mathbb{Z}[i]$.  In Section
\ref{sec:struct-ende-char-zero} we established that the elliptic curve $E$ given by
the equation
\begin{equation*}
  y^{2} = x^{3} + x
\end{equation*}
has complex multiplication by $\mathbb{Z}[i]$.  In particular, we know that $h_{k} =
1$ so $j(E)$ is rational (indeed, we have $j(E) = 1728$).

Consider the group $E[2]$ of $2$-torsion points of $E$.  Recall that a point $P =
(x,y)$ on $E$ has order $2$ if and only if $y = 0$.  So the $x$-co-ordinates are
given by the roots of the equation
\begin{equation*}
  x^{3} + x = 0,
\end{equation*}
which are clearly equal to $0$, $i$ and $-i$.  In particular, the field $K_{2} =
\mathbb{Q}(i)(E[2])$ is generated by elements contained in $\mathbb{Q}(i)$, so that
$K_{2} = \mathbb{Q}(i)$.

Now consider the group of $3$-torsion points of $E$.  There are $8$ non-trivial such
points (since $ \# E[3] = 9$), whose $x$-co-ordinates are determined by the roots of
the quartic equation
\begin{equation}
  \label{eq:3-torsion-polynomial}
  3x^{4} + 6x^{2} - 1.
\end{equation}
Solving the corresponding quadratic in $u = x^{2}$ we see that the roots of
\eqref{eq:3-torsion-polynomial} are $\pm \alpha, \pm
\frac{i}{\sqrt{3}\alpha}$\footnote{This follows from the method of ``Lagrange
  resultants''; a monic quadratic polynomial $x^{2} + px + q$ over an algebraically
  closed field $k$ factors as $(x - \alpha)(x - \beta)$ where $\alpha$ and $\beta$
  are roots of the polynomial in $k$. In particular, comparing coefficients gives the
  identities $\alpha + \beta = p$ and $\alpha \beta = q$, the second of which we use
  here.}, where
\begin{equation*}
  \alpha = \sqrt{ \frac{2\sqrt{3} - 3}{3} }.
\end{equation*}
To find the $y$-co-ordinates, note that
\begin{equation*}
  \alpha^{3} + \alpha = \alpha(\alpha^{2} + 1) = \frac{2\alpha}{\sqrt{3}},
\end{equation*}
and
\begin{equation*}
  (\frac{i}{\sqrt{3}\alpha})^{3} + \frac{i}{\sqrt{3}\alpha} = \frac{-2i}{3\alpha} =
  \frac{-4i}{\sqrt{27} (\frac{2\alpha}{\sqrt{3}}) } = \frac{-4i}{\sqrt{27}\beta^{2}},
\end{equation*}
where
\begin{equation*}
  \beta = \sqrt{ \frac{2\alpha}{\sqrt{3}} },
\end{equation*}
so that
\begin{equation}
  \label{eq:points-of-e[3]}
  E[3] = \{ \mathcal{O}, (\alpha,\pm \beta), (-\alpha, \pm \beta),
  (\frac{i}{\sqrt{3}\alpha}, \pm \frac{2 \sqrt{-i}}{\sqrt[4]{27}\beta} ),
  (-\frac{i}{\sqrt{3}\alpha}, \pm \frac{2 \sqrt{i}}{\sqrt[4]{27}\beta} )  \}.
\end{equation}
We claim that $K_{3} = \mathbb{Q}(i)(E[3]) = \mathbb{Q}(i,\beta)$, i.e. the
co-ordinates of the points in $E[3]$ can be described by rational functions of $i$
and $\beta$. Indeed, we have
\begin{align}
  \beta^{2}&= \frac{2\alpha}{\sqrt{3}} \label{eq:beta-powers-1}\\
  \beta^{4}&= \frac{4}{9}(2\sqrt{3} - 3) \label{eq:beta-powers-2}
\end{align}
so \eqref{eq:beta-powers-2} shows that $\sqrt{3}$ belongs to $\mathbb{Q}(i,\beta)$,
which in turn (along with \eqref{eq:beta-powers-1}) shows that $\alpha$ is also in
$\mathbb{Q}(i,\beta)$.  It remains to check that the $y$-co-ordinates of the points
$(\frac{i}{\sqrt{3}\alpha}, \pm \frac{2 \sqrt{-i}}{\sqrt[4]{27}\beta} )$ and
$(-\frac{i}{\sqrt{3}\alpha}, \pm \frac{2 \sqrt{i}}{\sqrt[4]{27}\beta} )$ are
contained in $\mathbb{Q}(i)$.  We recall the addition formulae for an elliptic curve
of the form $y^{2} = x^{3} + Ax + B$.  If $P_{1} = (x_{1},y_{1})$ and $P_{2} =
(x_{2},y_{2})$ are points on such a curve then the co-ordinates $x_{3},y_{3}$ of
$P_{3} = P_{1} + P_{2}$ are given by
\begin{equation}
  \label{eq:addition-law}
  x_{3} = \lambda^{2} - x_{1} - x_{2} \quad \text{and} \quad  y_{3} = -\lambda x_{3}
  - \nu,
\end{equation}
where
\begin{equation}
  \label{eq:eddition-law-2}
  \lambda = \frac{y_{2} - y_{1}}{x_{2} - x_{1}} \quad \text{and} \quad \nu =
  \frac{y_{1}x_{2} - y_{2}x_{1}}{x_{2} - x_{1}}.
\end{equation}
Now $E[3]$ is a group, and if we let $P_{1} = (\alpha, \beta)$ and $P_{2} = (-\alpha,
i\beta)$ in $E[3]$ it follows that $P_{1} + P_{2}$ is equal to
$(\frac{i}{\sqrt{3}\alpha}, \pm \frac{2 \sqrt{-i}}{\sqrt[4]{27}\beta} )$ or
$(-\frac{i}{\sqrt{3}\alpha}, \pm \frac{2 \sqrt{i}}{\sqrt[4]{27}\beta} )$\footnote{One
  sees this easily by exhausting all other possibilities.  For example, we cannot
  have $P_{1} + P_{2} = 2P_{3}$ since that would imply $P_{1} = P_{2}$, which is
  absurd.}.  In particular, their values are given by rational functions of $\alpha$,
$\beta$ and $i$ (by \eqref{eq:addition-law} and \eqref{eq:eddition-law-2}), and since
$\lambda$ belongs to $\mathbb{Q}(i)$ we have the desired result, namely that $K_{3} =
\mathbb{Q}(i,\beta)$.

Now, the minimal polynomial for $\beta$ over $\mathbb{Q}(i)$ is
\begin{equation}
  \label{eq:min-poly-beta}
  x^{8} + \frac{8}{3}x^{4} - \frac{16}{27},
\end{equation}
so that $[K_{3} : \mathbb{Q}(i) ] = 8$.  Since \eqref{eq:min-poly-beta} has such a
simple form, we are reduced to solving a quadratic equation (as we did with
\eqref{eq:3-torsion-polynomial}), and we find that the roots of
\eqref{eq:min-poly-beta} are
\begin{equation*}
  \pm \beta, \pm i \beta, \pm \frac{2 \sqrt{i}}{\sqrt[4]{27}\beta}, \pm \frac{2 \sqrt{-i}}{\sqrt[4]{27}\beta}
\end{equation*}
(cf. \eqref{eq:points-of-e[3]}).  So the $8$ embeddings $\sigma_{j} : \mathbb{Q}(i)
\rightarrow \mathbb{C}$, $j = 1 \ldots 8$, are determined uniquely by the value of
$\sigma_{j}(\beta)$.  Furthermore, we know by Theorem
\ref{thm:K_m-is-abelian-over-k(j(E))} that $K_{3}$ is a Galois extension of
$\mathbb{Q}(i)$, with $Gal(K_{3}/\mathbb{Q}(i))$ abelian.  Since the only abelian
groups of order 8 are $C_{2} \times C_{4}$ and $C_{8}$ it is easy to see that
$Gal(K_{3}/\mathbb{Q}(i))$ must be isomorphic to $C_{2} \times C_{4}$.


%%% Local Variables: 
%%% mode: latex
%%% TeX-master: "../main"
%%% End: 
