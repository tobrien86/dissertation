\newpage
\section{Complex multiplication - basic results}
\label{sec:compl-mult-basic}

In characteristic zero, the endomorphism ring an elliptic curve $E$ with complex
multiplication is isomorphic to an order in the ring of integers of some quadratic imaginary number field
$k$ (Theorem \ref{thm:structure-thm-for-End(E)}).  For simplicity, we will only
consider the case where $End(E)$ is isomorphic to the full ring of integers
$\mathfrak{o}_{k}$ in some quadratic imaginary field $k$, and we shall say that \emph{$E$
has complex multiplication by $\mathfrak{o}_{k}$}.  We denote by
$Ell(\mathfrak{o}_{k})$ the set of isomorphism classes of elliptic curves with
complex multiplication by $\mathfrak{o}_{k}$.  Note that, if $E$ has complex
multiplication by $\mathfrak{o}_{k}$ then there are exactly two ways in which
$End(E)$ can be embedded in $\mathbb{C}$.  We will always use the identification
determined by the following commutative diagram (replacing $E$ by $E_{\Lambda}$ for
some lattice $\Lambda$, which is possible by Uniformisation):
\begin{equation}
  \label{eq:o-identified-with-End(E)}
  \begin{split}
  \xymatrix{
    \mathbb{C} / \Lambda \ar[d]^{f} \ar[r]^{\phi_{\alpha}} & \mathbb{C} / \Lambda \ar[d]^{f}\\
    E_{\Lambda} \ar[r]^{[\alpha]}& E_{\Lambda}
  }
  \end{split}
\end{equation}
where $f$ is the isomorphism described in \eqref{eq:complex-eqn-for-ec}, and
$\phi_{\alpha}$ and $[\alpha]$ are the maps determined in Proposition
\ref{prop:complex-isogenies} and \eqref{eq:End(E)-in-complex} respectively.

For example, let $\Lambda = \mathbb{Z}[i]$.  Then
\begin{equation*}
  \wp(iz) = \frac{1}{(iz)^{2}} + \sum_{\omega \in \Lambda - 0}\frac{1}{(iz - \omega)^{2}} - \frac{1}{\omega^{2}},
\end{equation*}
and since $i\Lambda = \Lambda$ we may replace $\omega$ with $i \omega$ in the above
summation, so that
\begin{equation*}
  \wp(iz) = -\wp(z).
\end{equation*}
Similarly, the same method shows that
\begin{equation*}
  \wp^{\prime}(iz) = i\wp(z).
\end{equation*}
Now consider the elliptic curve $E_{\Lambda}$.  Then
\eqref{eq:o-identified-with-End(E)} says that the endomorphism $[i]$ is determined by
\begin{equation*}
  [i](x,y) = (-x,y).
\end{equation*}
Furthermore, using again the fact that $i\Lambda = \Lambda$ we have
\begin{equation*}
  g_{3}(\Lambda) = g_{3}(i\Lambda) = 140 \sum_{\omega \in \Lambda} \frac{1}{(i
    \omega)^{6}} = -g_{3}(\Lambda),
\end{equation*}
so that $g_{3}(\Lambda) = 0$ and $E_{\Lambda}$ is given by the equation
\begin{equation*}
  y^{2} = 4x^{3} - g_{2}x.
\end{equation*}
Therefore, $E_{\Lambda}$ is isomorphic to the elliptic curve $E$ given by
\begin{equation*}
  y^{2} = x^{3} + x,
\end{equation*}
since $j(E_{\Lambda}) = 1728 = j(E)$.  This justifies the description of the
endomorphism described in Example \ref{ex:cm-example}.

\subsection{Fractional ideals}
\label{sec:fractional-ideals}

We recall some basic results from algebraic number theory which we shall need in the
sequel.

Let $k$ be a number field, with ring of integers $\mathfrak{o}_{k}$.  A
\emph{fractional ideal} of $k$ is a non-zero $\mathfrak{o}_{k}$-module $\mathfrak{a}$
of $k$ which satisfies one of the following two equivalent conditions:
\begin{enumerate}[(i)]
\item $\mathfrak{a}$ is finitely generated,
\item there exists a non-zero element $a$ in $\mathfrak{o}_{k}$ such that
  $a\mathfrak{a} \subset \mathfrak{o}_{k}$.
\end{enumerate}
Every ideal of $\mathfrak{o}_{k}$ is obviously a fractional ideal of $k$, and we
refer to such an ideal as an \emph{integral ideal}.  The quotient $\mathfrak{o}_{k} /
\mathfrak{a}$ is finite, and we define the \emph{norm $N\mathfrak{a}$ of
  $\mathfrak{a}$} by
\begin{equation*}
  N\mathfrak{a} = \# ( \mathfrak{o}_{k} / \mathfrak{a} ).
\end{equation*}
In particular, if $k$ is quadratic, then $k = \mathbb{Q}(\sqrt{d})$ for some
square-free integer $d$ and every element $\alpha$ of $k$ is of the form
\begin{equation*}
  \alpha = a + b\sqrt{d}
\end{equation*}
where $a$ and $b$ are in $\mathbb{Q}$.  When $\alpha$ is an integer in $k$
(i.e. $\alpha \in \mathfrak{o}_{k}$), then
$a^{2} - db^{2}$ is a rational integer, and the principle ideal $(\alpha)$ of
$\mathfrak{o}_{k}$ satisfies
\begin{equation*}
  N(\alpha) = |a^{2} - db^{2}|.
\end{equation*}
  A fractional ideal is \emph{principal}
if it is of the form $c\mathfrak{o}_{k}$ for some $c$ in $k$.  If $\mathfrak{a}$ is a
fractional ideal of $k$, then we define $\mathfrak{a}^{-1}$ to be the set
\begin{equation*}
  \mathfrak{a}^{-1} = \{x \in k \colon x\mathfrak{a} \subset \mathfrak{o}_{k}\}.
\end{equation*}
While not obvious from the definition, the set $\mathfrak{a}^{-1}$ is a fractional
ideal, and the product $\mathfrak{a} \cdot \mathfrak{a}^{-1}$ is equal to
$\mathfrak{o}_{k}$.  The fractional ideals of $k$ form an abelian group with identity
element $\mathfrak{o}_{k}$.  The quotient of this group by the principle fractional
ideals is known as the \emph{class group} of $k$, and is denoted by $Cl(k)$.  It is
finite, and its order $h_{k}$ is known as the \emph{class number} of $k$.  Each ideal
class in $Cl(k)$ can be represented by an integral ideal.

In the case where $k$ is a quadratic imaginary field, the following result will allow
us to construct elliptic curves with complex multiplication by $\mathfrak{o}_{k}$.

\begin{lem}
  \label{lem:frac-ideals-are-lattices}
  Let $k$ be a quadratic imaginary number field.  Every fractional ideal
  $\mathfrak{a}$ of $k$ is a lattice in $\mathbb{C}$.
\end{lem}

\begin{proof}
  Let $\mathfrak{a}$ be a fractional ideal of $k$.  Using the inclusion $\mathbb{Z}
  \subset \mathfrak{o}_{k}$ it follows that $\mathfrak{a}$ is a free
  $\mathbb{Z}$-module of rank $2$.  Furthermore, it is clear that $\mathfrak{a}$ is
  not contained in $\mathbb{R}$. Thus $\mathfrak{a}$ is a lattice, as required.
\end{proof}

\subsection{Constructing elliptic curves with complex multiplication by
  $\mathfrak{o}_{k}$}
\label{sec:constr-ellipt-curv-with-cm-by-o}

Thus far we have not found a satisfactory way of constructing elliptic curves (over
$\mathbb{C}$) with complex multiplication.  We shall remedy this situation now.  More
precisely, to every fractional ideal $\mathfrak{a}$ of $k$, for some quadratic
imaginary field $k$, we will associate an elliptic curve which has complex
multiplication by $\mathfrak{o}_{k}$.  We shall see later (Theorem
\ref{thm:class-group-action-is-simply-transitive}) that every elliptic curve with
complex multiplication by $\mathfrak{o}_{k}$ is isomorphic to one which is obtained
from a fractional ideal of $k$.

\begin{thm}
  \label{thm:cm-curves-from-frac-ideals}
  Let $k$ be a quadratic imaginary field, $\mathfrak{o}_{k}$ its ring of integers,
  and let $\mathfrak{a}$ be a fractional ideal of $k$.  Then the elliptic curve
  $E_{\mathfrak{a}}$ has complex multiplication by $\mathfrak{o}_{k}$.
\end{thm}
\begin{proof}
  Consider the endomorphism ring of the elliptic curve $E_{\mathfrak{a}}$:
  \begin{equation*}
    End(E_{\mathfrak{a}}) \cong \{ \alpha \in \mathbb{C} \colon \alpha\mathfrak{a}
    \subset \mathfrak{a}\}
  \end{equation*}
  Now, since $\mathfrak{a} \subset k$, any such $\alpha$ must be an element of $k$.
  But $\mathfrak{a}$ is a finitely generated $\mathfrak{o}_{k}$-submodule of $k$, so
  in fact $\alpha$ must be an element in $\mathfrak{o}_{k}$.  Conversely, any element
  $\alpha$ in $\mathfrak{o}_{k}$ trivially satisfies $\alpha\mathfrak{a} \subset
  \mathfrak{a}$.  Thus $End(E_{\mathfrak{a}}) \cong \mathfrak{o}_{k}$.
\end{proof}
Recall that homothetic lattices give rise to isomorphic elliptic curves by Corollary
\ref{cor:homothetic-lattices-give-isomorphic-curves}.  In particular, replacing a
fractional ideal $\mathfrak{a}$ of $k$ with $c\mathfrak{a}$ for some non-zero element
$c$ in $k$ will give an elliptic curve $E_{c\mathfrak{a}}$ which has complex
multiplication by $\mathfrak{o}_{k}$ and is isomorphic to $E_{\mathfrak{a}}$.  This
suggest that the class group $Cl(k)$ of $k$ may play a role in the theory.

Let $\Lambda$ be a lattice in $\mathbb{C}$.  For a fractional ideal $\mathfrak{a}$ of
$k$ we define the product $\mathfrak{a}\Lambda$ to be the subset of $\mathbb{C}$
consisting of finite sums of products of elements of $\mathfrak{a}$ and $\Lambda$:
\begin{equation*}
  \mathfrak{a}\Lambda = \{ \sum x_{i}\omega_{i} \colon x_{i} \in \mathfrak{a},
  \omega_{i} \in \Lambda\}.
\end{equation*}

\begin{lem}
  \label{lem:o.lambda-equals-lamba}
  Suppose $\Lambda$ is such that the elliptic curve $E_{\Lambda}$ has complex
  multiplication by $\mathfrak{o}_{k}$.  Then $\mathfrak{o}_{k}\Lambda = \Lambda$,
  and $\mathfrak{a}\Lambda$ is a lattice in $\mathbb{C}$ for every fractional ideal
  $\mathfrak{a}$ of $k$.
\end{lem}
\begin{proof}
  The first part is clear, since $End(E_{\Lambda}) \cong \{\alpha \in \mathbb{C}
  \colon \alpha\Lambda \subset \Lambda \}$.  Now let $\mathfrak{a}$ be a fractional
  ideal of $k$.  Since $k$ is a quadratic field, we have $k = \mathbb{Q}(\sqrt{d})$
  for some squarefree integer $d$, and
  \begin{equation*}
    \mathfrak{o}_{k} =
    \begin{cases}
      \mathbb{Z}[\sqrt{d}] & d \equiv 2 \text{ or } 3 \pmod{4} ,\\
      \mathbb{Z}[\frac{1 + \sqrt{d}}{2}] & d \equiv 1 \pmod{4}.
    \end{cases}
  \end{equation*}
  Let $a$ be an element of $\mathfrak{o}_{k}$ such that $a\mathfrak{a} \subset
  \mathfrak{o}_{k}$.  We may assume $a$ is a rational integer; if $d \equiv 2 \text{
    or } 3 \pmod{4}$ then $a = x + y\sqrt{d}$ for some rational integers $x$ and $y$,
  and if $a$ is not rational we may replace it with $(x - y\sqrt{d})\cdot a$ which is
  a rational integer taking $\mathfrak{a}$ into $\mathfrak{o}_{k}$.  Similarly if $d
  \equiv 1 \pmod{4}$ then $a = \frac{x + y\sqrt{d}}{2}$, and replacing $a$ with
  $4\cdot (\frac{x - y\sqrt{d}}{2})\cdot a$ gives the desired result.  Since
  $a\mathfrak{a} \subset \mathfrak{o}_{k}$ the first part of the lemma shows
  \begin{equation*}
    \mathfrak{a}\Lambda \subset \frac{1}{a}\Lambda,
  \end{equation*}
  so $\mathfrak{a}\Lambda$ is a discrete subgroup of $\mathbb{C}$ and is hence a
  lattice by Lemma \ref{lem:latice-iff-discrete}.
\end{proof}

Suppose $\Lambda$ is a lattice such that $E_{\Lambda}$ has complex multiplication by
$\mathfrak{o}_{k}$.  Lemma \ref{lem:o.lambda-equals-lamba} allows us to construct the
elliptic curve $E_{\mathfrak{a}\Lambda}$, for some fractional ideal $\mathfrak{a}$ of
$k$.  The endomorphism ring of $E_{\mathfrak{a}\Lambda}$ is isomorphic to
\begin{equation*}
  \{\alpha \in \mathbb{C} : \alpha \mathfrak{a}\Lambda \subset \mathfrak{a}\Lambda \}.
\end{equation*}
However, since $\mathfrak{a} \cdot \mathfrak{a}^{-1} = \mathfrak{o}_{k}$ we have
\begin{equation*}
  \alpha \mathfrak{a}\Lambda \subset \mathfrak{a}\Lambda
  \Leftrightarrow
  \alpha\Lambda \subset \Lambda ,
\end{equation*}
so $E_{\mathfrak{a}\Lambda}$ has complex multiplication by $\mathfrak{o}_{k}$.

It is now easy to see that there is a well-defined action of the class group $Cl(k)$
on $Ell(\mathfrak{o}_{k})$.  For an ideal class $\mathfrak{a}$ in $Cl(k)$ and an
elliptic curve $E_{\Lambda}$ in $Ell(\mathfrak{o}_{k})$ we define the action as
follows:
\begin{equation*}
  E_{\Lambda}^{\mathfrak{a}} = E_{\mathfrak{a}^{-1}\Lambda}.
\end{equation*}
That the action is well-defined follows from the remarks after Theorem
\ref{thm:cm-curves-from-frac-ideals}.  Furthermore, we have
\begin{equation*}
  E_{\Lambda}^{\mathfrak{o}_{k}} = E_{\mathfrak{o}^{-1}\Lambda} =
  E_{\mathfrak{o}\Lambda} = E_{\Lambda},
\end{equation*}
and
\begin{equation*}
  (E_{\Lambda}^{\mathfrak{a}})^{\mathfrak{b}} =
  E_{\mathfrak{a}^{-1}\Lambda}^{\mathfrak{b}} =
  E_{\mathfrak{b}^{-1}\mathfrak{a}^{-1}\Lambda} = E_{\mathfrak{ab}^{-1}\Lambda} = E_{\Lambda}^{\mathfrak{ab}},
\end{equation*}
where $\mathfrak{a}$ and $\mathfrak{b}$ represent any non-trivial ideal classes in
$Cl(k)$.
\begin{rem}
  \label{rem:convention-on-group-action}
  We choose $\mathfrak{a}^{-1}$ in the definition of the action out of respect for
  convention.  The use of the inverse makes things easier in the deeper theory of
  complex multiplication (which we will not quite reach).
\end{rem}

\begin{lem}
  \label{lem:Q-adjoin-w1-over-w2-is-quadratic-imaginary}
  Let $\Lambda$ be a lattice such that $E_{\Lambda}$ has complex multiplication by
  $\mathfrak{o}_{k}$.  Let $(\omega_{1},\omega_{2})$ be a generating set for
  $\Lambda$ over $\mathbb{Z}$.  Then $\mathbb{Q}(\frac{\omega_{1}}{\omega_{2}})$ is a
  quadratic imaginary field, equal to $k$.
\end{lem}
\begin{proof}
  Let $\tau = \frac{\omega_{1}}{\omega_{2}}$.  In particular $\tau$ is non-real, by
  definition of $\Lambda$.  We may assume that $\Lambda = \mathbb{Z} +
  \mathbb{Z}\tau$, since homothetic lattices give rise to isomorphic curves.  Let $R
  = \{\alpha \in \mathbb{C} : \alpha \Lambda \subset \Lambda \} \cong
  End(E_{\Lambda}) \}$.  Then, since both $1$ and $\tau$ belong to $\Lambda$, there
  exist rational integers $a$, $b$, $c$ and $d$ such that
  \begin{equation*}
    \alpha = a + b\tau \quad \text{and} \quad \alpha \tau = c + d \tau,
  \end{equation*}
  for any $\alpha$ in $R$ (so $R \subset \Lambda)$.  Re-arranging the above equations
  yields
  \begin{equation*}
    b\tau^{2} + a\tau = \alpha\tau = c + d\tau,
  \end{equation*}
  so that $\tau$ is a non-real root of an integral quadratic equation, and hence the
  field $\mathbb{Q}(\tau) = \mathbb{Q}(\frac{\omega_{1}}{\omega_{2}})$ is quadratic
  imaginary.  Similary, we have
  \begin{equation*}
    (\alpha - a)(\alpha - d) = (\alpha - d)b\tau = bc,
  \end{equation*}
  so every $\alpha$ in $R \subset \mathbb{Q}(\tau)$ is a root of a monic integral
  quadratic equation, and hence is an integer in $\mathbb{Q}(\tau)$.  This shows that
  $R \otimes \mathbb{Q} = \mathbb{Q}(\tau)$.  On the other hand, we have $R \cong
  End(E)$ so $R \otimes \mathbb{Q} = k$.  Hence
  $\mathbb{Q}(\frac{\omega_{1}}{\omega_{2}}) = k$, as required.
\end{proof}

\begin{thm}
  \label{thm:class-group-action-is-simply-transitive}
  The action of $Cl(k)$ on $Ell(\mathfrak{o}_{k})$ is simply transitive.  In
  particular,
  \begin{equation*}
    \# Ell(\mathfrak{o}_{k}) \leq h_{k}.
  \end{equation*}
\end{thm}
\begin{proof}
  Let $\Lambda_{1}$ and $\Lambda_{2}$ be lattices in $\mathbb{C}$ such that the
  elliptic curves $E_{\Lambda_{1}}$ and $E_{\Lambda_{2}}$ have complex multiplication
  by $\mathfrak{o}_{k}$.  Choose a non-zero element $\lambda_{1}$ in $\Lambda_{1}$
  and consider the lattice $\mathfrak{a}_{1} = \frac{a}{\lambda_{1}}\Lambda$.  Let
  $(\omega_{1},\omega_{2})$ be a generating set for $\Lambda_{1}$ over $\mathbb{Z}$.
  Then $\lambda_{1} = c\omega_{1} + d\omega_{2}$ for some rational integers $c$ and
  $d$, and the elements of $\mathfrak{a}_{1}$ are of the form
  \begin{equation*}
    \frac{a\omega_{1} + b\omega_{2}}{c\omega_{1} + d\omega_{2}}.
  \end{equation*}
  Multiplying the numerator and denominator by $\frac{1}{\omega_{2}}$ shows that
  $\mathfrak{a}_{1}$ is contained in $k$, by Lemma
  \ref{lem:Q-adjoin-w1-over-w2-is-quadratic-imaginary}.  Furthermore, since
  $\mathfrak{o}_{k} \cong E = \{\alpha \in \mathbb{C} : \alpha \Lambda_{1} \subset
  \Lambda_{1} \}$ we see that there is a homomorphism $\mathfrak{o}_{k} \rightarrow
  End(\mathfrak{a}_{1})$ determined by the following diagram:
  \begin{equation*}
    \xymatrix{
      \mathfrak{o}_{k} \ar[rd] \ar[r] & E \ar[d]^{f}\\
      &End(\mathfrak{a}_{1})
    }
  \end{equation*}
  where $f(\alpha)$ is the endomorphism of $\mathfrak{a}_{1}$ given by
  \begin{equation*}
    f(\alpha)(x) = \alpha \cdot x.
  \end{equation*}
  Thus $\mathfrak{a}_{1}$ is a finitely-generated $\mathfrak{o}_{k}$-module, and is
  hence a fractional ideal of $k$.  Similarly, choosing a non-zero element
  $\lambda_{2}$ of $\Lambda_{2}$ gives a fractional ideal $\mathfrak{a}_{2} =
  \frac{1}{\lambda_{2}}$ of $k$.  We have
  \begin{equation*}
    \frac{\lambda_{2}}{\lambda_{1}}\mathfrak{a}_{2}\mathfrak{a}_{1}^{-1}\Lambda_{1} = \Lambda_{2},
  \end{equation*}
  so that
  \begin{equation*}
    E_{\Lambda_{1}}^{\mathfrak{a}_{2}^{-1}\mathfrak{a}_{1}} =
    E_{\mathfrak{a}_{2}\mathfrak{a}_{1}^{-1}\Lambda_{1}} = E_{\frac{\lambda_{1}}{\lambda_{2}}\Lambda_{2}},
  \end{equation*}
  and $E_{\frac{\lambda_{1}}{\lambda_{2}}\Lambda_{2}}$ is isomorphic to
  $E_{\Lambda_{2}}$ by Corollary
  \ref{cor:homothetic-lattices-give-isomorphic-curves}.  Thus $Cl(k)$ acts
  transitively on $Ell(\mathfrak{o}_{k})$.  It remains to prove that the action is
  simply transitive, i.e. if $E_{\Lambda}^{\mathfrak{a}} =
  E_{\Lambda}^{\mathfrak{b}}$ then $\mathfrak{a} = \mathfrak{b}$ in $Cl(k)$.  So,
  suppose $E_{\Lambda}^{\mathfrak{a}} = E_{\Lambda}^{\mathfrak{b}}$.  Then 
\end{proof}

\subsection{The group of $\mathfrak{a}$-torsion points}
\label{sec:group-a-tors}

In section \ref{sec:some-properties-ende} we determined the structure of the group
$E[m]$ of $m$-torsion points of an arbitrary elliptic curve $E$.  In the case where
$E$ has complex multiplication by $\mathfrak{o}_{k}$ it will be helpful to consider
other subgroups of $E$.

Let $E$ be an elliptic curve with complex multiplication by $\mathfrak{o}_{k}$, and
let $\mathfrak{a}$ be an integral ideal of $k$.  Let $\Lambda$ be a lattice in
$\mathbb{C}$ such that $E \cong E_{\Lambda}$.  By Lemma
\ref{lem:o.lambda-equals-lamba} we have
\begin{equation*}
  \mathfrak{a}\Lambda \subset \Lambda,
\end{equation*}
so that
\begin{equation*}
  \Lambda \subset \mathfrak{a}^{-1}\Lambda,
\end{equation*}
which induces an isogeny $\phi_{\mathfrak{a}} : E_{\Lambda} \rightarrow
E_{\Lambda}^{\mathfrak{a}}$ by (TODO: put in isogeny stuff in complex review):
\begin{equation*}
  \xymatrix{
    \mathbb{C} / \Lambda \ar[d] \ar[r] & \mathbb{C} / \mathfrak{a}^{-1}\Lambda \ar[d] \\
    E_{\Lambda} \ar[r]^{\phi_{\mathfrak{a}}} & E_{\Lambda}^{\mathfrak{a}}.
  }
\end{equation*}
\begin{prop}
  \label{prop:a-torsion-is-kernel}
  The kernel of the isogeny $\phi_{\mathfrak{a}}$ is the set
  \begin{equation*}
    \ker{\phi_{\mathfrak{a}}} = \{ P \in E : [\alpha]P = \mathcal{O} \text{ for all } \alpha
    \in \mathfrak{a} \},
  \end{equation*}
  where $[\alpha]$ is the endomorphism of $E$ determined by
  \eqref{eq:o-identified-with-End(E)}.  Furthermore, it is a free $\mathfrak{o}_{k} /
  \mathfrak{a}$-module of rank $1$.
\end{prop}
\begin{proof}
  See (\cite{silverman94} II, Prop. 1.4).
\end{proof}

We thus define the kernel of $\phi_{\mathfrak{a}}$ to be the \emph{group of
  $\mathfrak{a}$-torsion points of $E$} and denote it by $E[\mathfrak{a}]$.

%%% Local Variables: 
%%% mode: latex
%%% TeX-master: "../main"
%%% End: 
