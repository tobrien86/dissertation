\section{The endomorphism ring of an elliptic curve}
\label{sec:endomorphism-ring-of-an-elliptic-curve}

Unless stated otherwise, all elliptic curves are defined over a field $k$ of
characteristic not equal to $2$ or $3$.

\subsection{Isogenies}
\label{sec:isogenies}

Let $E_{1}$ and $E_{2}$ be elliptic curves.  An \emph{isogeny} from $E_{1}$ to
$E_{2}$ is a morphism (of projective varieties) $\phi \colon E_{1} \rightarrow E_{2}$
which satisfies
\begin{equation*}
  \phi \left( O \right) = O.
\end{equation*}
Any isogeny is automatically a group homomorphism (\cite{silverman86} III, \S 4, Thm
4.8).  Furthermore, a non-zero isogeny is surjective (\cite{shafarevich94} I, \S 5.3,
Thm 4).  Therefore, such an isogeny $\phi \colon E_{1} \rightarrow E_{2}$ induces an
injective homomorphism of function fields $\phi^{*} \colon \bar{k}\left(E_{2}\right)
\rightarrow \bar{k}\left(E_{1}\right)$ given by
\begin{equation*}
  \phi^{*}\left(f\right) = f \circ \phi.
\end{equation*}
The extension $\bar{k}\left(E_{1}\right) /
\phi^{*}\left(\bar{k}\left(E_{2}\right)\right)$ is finite (\cite{hartshorne77} II,
Prop 6.8), and we thus define the \emph{degree} of $\phi$ to be the degree of the
field extension.  We define the degree of the zero isogeny to be zero.  If $\phi_{1}
\colon E_{1} \rightarrow E_{2}$ and $\phi_{2} \colon E_{2} \rightarrow E_{3}$ are
isogenies of elliptic curves, then $\phi_{2} \circ \phi_{1}$ is an isogeny, where
\begin{equation}
  \label{eq:degree-of-composition}
  deg \left( \phi_{2} \circ \phi_{1} \right) = deg \left( \phi_{2} \right) deg \left( \phi_{1} \right).
\end{equation}
We say $\phi$ is \emph{separable}, \emph{inseparable} or \emph{purely inseparable}
according to the extension.

An \emph{endomorphism} of an elliptic curve $E$ is an isogeny from $E$ to itself.
The set of all endomorphisms of $E$ forms a ring $End(E)$ under pointwise addition
and composition of morphisms, and is known as the \emph{endomorphism ring} of $E$.

\subsection{Some properties of $End(E)$}
\label{sec:some-properties-ende}

We will show that the endomorphism ring of an elliptic curve has a very particular
structure.  The following example allows us to determine some basic properties.

\begin{example}
  \label{ex:mult-by-m}
  Let $E$ be an elliptic curve.  For every rational integer $m$ the \emph{multiplication-by-m}
  map $\left[m\right] \colon E \rightarrow E$ defined by
%  \begin{equation*}
%    \left[m\right]P = P + \ldots + P \qquad \text{($m$ summands)}
%  \end{equation*}
  \begin{equation*}
    \left[ m \right] P =
    \begin{cases}
      \quad O & m = 0,\\
      \quad P + \ldots + P & m > 0,\\
      - \left( P + \ldots + P \right) & m < 0,
    \end{cases}
  \end{equation*}
  is an endomorphism of $E$.  Its kernel is the familiar subgroup $E \left[ m
  \right]$ of $E$.  From the definition it follows that $\left[m\right]
  \circ \left[n\right] = \left[mn\right]$.  
  %%% Commented the following because it's obvious, though may be useful as a space
  %%% filler.
  % Indeed, we have in general that if $P_{1} =
  % \left(x_{1},y_{1}\right)$ and $P_{2} = \left(x_{2},y_{2}\right)$ are points on
  % $E$,
  % then
\end{example}

An elliptic curve $E$ has precisely three points of order 2.  Since $E$ is infinite,
there exists a point $P_{1}$ on $E$ of order not equal to 2 so that
$\left[2\right]P_{1} \neq O$.  Similarly, for a point $P_{2}$ of order 2, any odd
integer $m$ satisfies $\left[m\right]P_{2} = P_{2}$.  Thus for all non-zero $m$ it
follows that $\left[m\right] \neq \left[0\right]$.

As a $\mathbb{Z}$-module, the endomorphism ring of an elliptic curve $E$ is torsion-free; if $\phi \in
End(E)$ and $m \in \mathbb{Z}$ satisfy
\begin{equation*}
  \left[m\right] \circ \phi = \left[ 0 \right]
\end{equation*}
then by \eqref{eq:degree-of-composition},
\begin{equation*}
  deg \left(\left[m\right]\right) \cdot deg \left( \phi \right) = 0
\end{equation*}
whence either $m = 0$ or $deg\left(\left[m\right]\right) > 0$, in which case $\phi =
\left[0\right]$ and $m \neq 0$.

Taking degrees also shows that $End(E)$ is an integral domain; if $\phi_{1}$ and
$\phi_{2}$ are endomorphisms of $E$ such that
\begin{equation*}
  \phi_{2} \circ \phi_{2} = \left[0\right],
\end{equation*}
then
\begin{equation*}
  deg \left( \phi_{2} \right) \cdot deg \left( \phi_{2} \right) = 0,
\end{equation*}
from which the result follows.

We summarise what we have shown in the following proposition.

\begin{prop}
  \label{prop:End(E)-is-char-zero-id}
  Let $E$ be an elliptic curve.  Then $End(E)$ is a characteristic zero integral
  domain.
\end{prop}

An elliptic curve whose endomorphism ring is strictly larger than $\mathbb{Z}$ is
said to have \emph{complex multiplication}.

\begin{example}
  \label{ex:cm-example}
  Consider the elliptic curve $E$ given by the equation
  \begin{equation*}
    y^{2} = x^{3} + x.
  \end{equation*}
  The map $\left[ i \right] \colon E \rightarrow E$ given by
  \begin{equation*}
    \left[ i \right](x,y) = (-x,iy)
  \end{equation*}
  is an endomorphism of $E$.  Note that $\left[ i \right]^{2} = \left[ -1 \right]$,
  so that $\left[ i \right] \neq \left[ m \right]$ for any rational integer $m$.
  Thus $E$ has complex multiplication.
  % Worked example showing End(E) = Z[i]

  We will see shortly that the endomorphism ring of any elliptic curve with complex
  multiplication has the structure of an order in an imaginary quadratic field
\end{example}

\subsection{An interlude on dual isogenies}
\label{sec:an-interlude-dual}

Let $E_{1}$ and $E_{2}$ be elliptic curves.  For every non-zero isogeny $\phi \colon
E_{1} \rightarrow E_{2}$ there exists a unique isogeny $\bar{\phi} \colon E_{2}
\rightarrow E_{1}$ which satisfies
\begin{equation}
  \label{eq:dual-isogeny}
  \bar{\phi} \circ \phi = \left[ m \right],
\end{equation}
where $m$ is the degree of $\phi$ (when $\phi = \left[ 0 \right]$ we define
$\bar{\phi}$ to be $\left[ 0 \right]$).  We say $\bar{\phi}$ is the \emph{dual
  isogeny} to $\phi$.

%% Some properties of dual isogenies here, for filler mostly. No proofs, just refs.

