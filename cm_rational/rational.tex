\newpage
\section{Complex multiplication over algebraic extensions of $\mathbb{Q}$}
\label{sec:compl-mult-over-Q}

TODO: intro.
nn
nLet $E$ be an elliptic curve defined over $\mathbb{C}$, given by the equation
\begin{equation*}
  y^{2} = x^{3} + Ax + B, \quad A, B \in \mathbb{C},
\end{equation*}
and let $\sigma$ be a field automorphism of $\mathbb{C}$. We define $E^{\sigma}$ to
be the elliptic curve obtained by letting $\sigma$ act on the coefficients of the
equation of $E$, i.e. $E^{\sigma}$ is obtained from the equation
\begin{equation*}
  y^{2} = x^{3} + A^{\sigma}x + B^{\sigma}.
\end{equation*}
It is clear that $E^{\sigma} = \{ P^{\sigma} : P \in E \}$, for if $P = (x,y)$, then
$P^{\sigma} = (x^{\sigma},y^{\sigma})$ satisfies
\begin{equation*}
  (y^{\sigma})^{2} - ( (x^{\sigma})^{3} + A^{\sigma}(x^{\sigma}) + B^{\sigma} ) = (y^{2} -
  (x^{3} + Ax + B))^{\sigma} = 0^{\sigma} = 0.
\end{equation*}
Note that, since $j(E)$ is given by a rational function in $A$ and $B$ we have
\begin{equation}
  \label{eq:j(E-sigma)-equals-j(E)-sigma}
  j(E^{\sigma}) = j(E)^{\sigma}
\end{equation}
Now, if $\phi$ is an endomorphism of $E$, then $\phi$ is given by rational maps in
the function field $\mathbb{C}(E)$:
\begin{equation*}
  \phi = (f_{1},f_{2}), \quad f_{i} \in \mathbb{C}(E),
\end{equation*}
and we define an endomorphism $\phi^{\sigma}$ of $E^{\sigma}$ in the obvious way:
\begin{equation*}
  \phi^{\sigma} = (f_{1}^{\sigma},f_{2}^{\sigma}).
\end{equation*}
Note that, if $(x\prime{},y\prime{})$ TODO: make this bit not messy.

Thus for an elliptic curve $E$ and an automorphism $\sigma$ of $\mathbb{C}$ we have
shown that
\begin{equation}
  \label{eq:End(E)-equals-End(E-sigma)}
  End(E) \cong End(E^{\sigma}).
\end{equation}

\begin{prop}
  \label{prop:j(E)-is-in-Qbar}
  Let $E$ be an elliptic curve with complex multiplication by $\mathfrak{o}_{k}$.
  Then $j(E)$ is algebraic over $\mathbb{Q}$.
\end{prop}
\begin{proof}
  Consider the extension $K = \mathbb{Q}(j(E))$ of $\mathbb{Q}$.  By Theorem
  \ref{thm:class-group-action-is-simply-transitive} we know that there only finitely
  many values that the $j$-invariant of an elliptic curve with complex multiplication
  by $\mathfrak{o}_{k}$ can attain (recall that the $j$-invariant classifies elliptic
  curves up to isomorphism), and by \eqref{eq:j(E-sigma)-equals-j(E)-sigma} and
  \eqref{eq:End(E)-equals-End(E-sigma)} we have that the set
  \begin{equation*}
    \{ j(E)^{\sigma} : \sigma \in Aut(\mathbb{C)} \}
  \end{equation*}
  is finite.  Now every $K$-embedding $\sigma : K \rightarrow \mathbb{C}$ is
  determined by $\sigma (j(E))$.  On the other hand, any such $\sigma$ is the
  restriction of an automorphism of $\mathbb{C}$.  Thus, there are only finitely many
  such $\sigma$, so that $[\mathbb{Q}(j(E)) : \mathbb{Q}]$ is finite, which implies
  that $j(E)$ is algebraic over $\mathbb{Q}$.
\end{proof}
\begin{rem}
  \label{rem:j(E)-is-integral}
  There is a much stronger result, which states that for an elliptic curve $E$ with
  complex multiplication by $\mathbb{O}_{k}$, we have that $j(E)$ is an algebraic
  integer.
  TODO: give reference/do proof later.
\end{rem}

Let $Ell_{\bar{\mathbb{Q}}}(\mathfrak{o}_{k})$ denote the set of isomorphism classes of
elliptic curves defined over $\bar{\mathbb{Q}}$.  If $E$ is an elliptic curve defined
over $\bar{\mathbb{Q}}$ then its isomorphism class (in
$Ell_{\bar{\mathbb{Q}}}(\mathfrak{o}_{k})$) is certainly contained in the
isomorphism class of $E$ considered as an elliptic curve over $\mathbb{C}$. Let
$i :
Ell_{\bar{\mathbb{Q}}}(\mathfrak{o}_{k}) \rightarrow Ell(\mathfrak{o}_{k})$ denote
this map.

\begin{thm}
  \label{thm:Ell-Q-(o_k)-equals-Ell(o_k)}
  The map $i :
Ell_{\bar{\mathbb{Q}}}(\mathfrak{o}_{k}) \rightarrow Ell(\mathfrak{o}_{k})$ is a bijection.
\end{thm}
\begin{proof}
  By Proposition \ref{prop:j(E)-is-in-Qbar}, any elliptic curve $E$ with complex
  multiplication by $\mathfrak{o}_{k}$ satisfies $j(E) \in \bar{\mathbb{Q}}$, and
  from the general theory of elliptic curves we know that $E$ is isomorphic to a
  curve defined over $\mathbb{Q}(j(E))$, given by the equation:
  \begin{equation}
    \label{E-defined-over-k(j(E))}
    TODO.
  \end{equation}
  
  so $i$ is surjective.  Similarly, if $E_{1}$ and $E_{2}$ are elliptic curves over
  $\bar{\mathbb{Q}}$ with complex multiplication by $\mathfrak{o}_{k}$, such that
  $i(E_{1}) = i(E_{2})$, then $j(E_{1}) = j(E_{2})$, so the curves must be
  isomorphic.  Hence $i$ is injective, as required.
\end{proof}
The above theorem justifies our approach to an essentially algebraic phenomenon via
analytic methods.  From an arithmetic point of view the set
$Ell_{\bar{\mathbb{Q}}}(\mathfrak{o}_{k})$ is ``more interesting'' than
$Ell(\mathfrak{o}_{k})$, but Theorem \ref{thm:Ell-Q-(o_k)-equals-Ell(o_k)} says that
the latter set, which is easily computed, will suit our purposes entirely.

\subsection{Towards abelian extensions of quadratic imaginary fields}
\label{sec:towards-abel-extens}

TODO: introduction.

Let $E$ be an elliptic curve defined over $\mathbb{C}$ with complex multiplication by
$\mathfrak{o}_{k}$.  For every positive integer $m$ let $K_{m}$ be the field
extension of $K$ obtained by adjoining to $K$ the value $j(E)$ and the co-ordinates
of the (non-trivial) $m$-torsion points of $E$ (c.f. section
\ref{sec:some-properties-ende}):
\begin{equation}
  \label{eq:defn-of-K_m}
  K_{m} = k(j(E),x_{1},y_{1},\ldots,x_{r},y_{r})
\end{equation}
where $(x_{i},y_{i})$ are the co-ordinates of the points in $E[m]$.  To ease notation
we write
\begin{equation*}
  K_{m} = k(j(E),E[m]).
\end{equation*}

\begin{lem}
  \label{lem:m-torsion-points-are-algebraic}
  TODO: this.
\end{lem}
\begin{proof}
  TODO: this.
\end{proof}

\begin{thm}
  \label{thm:K_m-is-abelian-over-k(j(E))}
  The field $K_{m}$ is an abelian extension of $k(j(E))$.
\end{thm}
\begin{proof}
  We need to show that $K_{m}$ is a Galois extension of $k$, and its Galois group is
  abelian.  Since $k$ is of characteristic zero, we only need to show that
  $\sigma(K_{m}) = K_{m}$ for any $K_{m}$-embedding $\sigma$.  By Lemma
  \ref{lem:m-torsion-points-are-algebraic} we know that $K_{m}$ is a finite extension
  of $k$, so any $K_{m}$-embedding is obtained from an element $\sigma$ in
  $Gal(\bar{k} / k)$.  Then by (TODO: fix reference), we have that $\sigma(x_{i}) =
  x_{j}$ and $\sigma(y_{i}) = y_{j}$ for all points $(x_{i},y_{i})$ in $E[m]$.  So
  $K_{m}$ is a normal extension of $k$, and hence is Galois.

  To see that $Gal(K_{m}/k)$ is abelian is more subtle.  Recall the representation of
  $Gal(\bar{k} / k)$ given in \eqref{eq:Galois-representation}:
  \begin{equation*}
    \rho (\sigma) (P) = P^{\sigma}.
  \end{equation*}
  We claim that $\rho$ is injective.  Indeed, suppose that $\sigma \in \ker{\rho}$.
  Then $P^{\sigma} = P$ for every $P$ in $E[m]$.  In particular, we have
  \begin{equation*}
    \sigma (x_{i}) = x_{i} \quad \text{and} \quad \sigma (y_{i}) = y_{i},
  \end{equation*}
  for every $P = (x_{i},y_{i})$ in $E[m]$.  So $\sigma$ is the identiy on $K_{m}$,
  and hence $\ker{\rho}$ is trivial, as required.

  Recall that $E[m]$ has the structure of an $\mathfrak{o}_{k} / m
  \mathfrak{o}_{k}$-module by Proposition \ref{prop:a-torsion-is-kernel}.
  Furthermore, we may assume $E$ is defined over $k(j(E))$ (by
  \eqref{E-defined-over-k(j(E))}), so that any endomorphism of $E$ is defined over
  $k(j(E))$, i.e. given by rational functions with coefficients in $k(j(E))$.  In
  particular, we have
  \begin{equation*}
    ([\alpha](P))^{\sigma} = [\alpha]^{\sigma}(P^{\sigma}) = [\alpha](P^{\sigma}),
    \quad [\alpha] \in \mathfrak{o}_{k}.
  \end{equation*}
  So $Gal(K_{m}/k(j(E)))$ injects into a subgroup of the $\mathfrak{o}_{k} / m
  \mathfrak{o}_{k}$-module automorphisms of $E[m]$, but by Proposition
  \ref{prop:a-torsion-is-kernel} we have
  \begin{equation*}
    Aut_{\mathfrak{o}_{k} / m \mathfrak{o}_{k}}(E[m]) = (\mathfrak{o}_{k} / m \mathfrak{o}_{k})^{*},
  \end{equation*}
  thus completing the proof.
\end{proof}
\begin{rem}
  \label{rem:not-abelian-over-k}
  In general the extension $K_{m}$ is \emph{not} an abelian extension of $k$.  Of
  course, if $j(E)$ is rational then $K_{m}$ is an abelian extension of $k$.  We
  illustrate this explicitly in the next section.
\end{rem}



%%% Local Variables: 
%%% mode: latex
%%% TeX-master: "../main"
%%% End: 
