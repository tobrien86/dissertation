\newpage
\section{The endomorphism ring of an elliptic curve}
\label{sec:endomorphism-ring-of-an-elliptic-curve}

Unless stated otherwise, all elliptic curves are defined over a field $k$ of
characteristic zero.

\subsection{Isogenies}
\label{sec:isogenies}

Let $E_{1}$ and $E_{2}$ be elliptic curves.  An \emph{isogeny} from $E_{1}$ to
$E_{2}$ is a morphism (of projective varieties) $\phi \colon E_{1} \rightarrow E_{2}$
which satisfies
\begin{equation*}
  \phi \left( \mathcal{O} \right) = \mathcal{O},
\end{equation*}
where $\mathcal{O}$ is the neutral element of $E$.  Any isogeny is automatically a
group homomorphism (\cite{silverman86} III, \S 4, Thm 4.8), and we denote by
$Hom(E_{1},E_{2})$ the group of all isogenies $\phi : E_{1} \rightarrow E_{2}$.  Any
non-zero isogeny is surjective (\cite{shafarevich94} I, \S 5.3, Thm 4).  Therefore,
an isogeny $\phi \colon E_{1} \rightarrow E_{2}$ induces an injective homomorphism of
function fields $\phi^{*} \colon \bar{k}\left(E_{2}\right) \rightarrow
\bar{k}\left(E_{1}\right)$ given by
\begin{equation*}
  \phi^{*}\left(f\right) = f \circ \phi.
\end{equation*}
The extension $\bar{k}\left(E_{1}\right) /
\phi^{*}\left(\bar{k}\left(E_{2}\right)\right)$ is finite (\cite{hartshorne77} II,
Prop 6.8), and we thus define the \emph{degree} of $\phi$ to be the degree of the
field extension.  We define the degree of the zero isogeny to be zero.  If $\phi_{1}
\colon E_{1} \rightarrow E_{2}$ and $\phi_{2} \colon E_{2} \rightarrow E_{3}$ are
isogenies of elliptic curves, then $\phi_{2} \circ \phi_{1}$ is an isogeny, and
\begin{equation}
  \label{eq:degree-of-composition}
  deg \left( \phi_{2} \circ \phi_{1} \right) = deg \left( \phi_{2} \right) deg \left( \phi_{1} \right),
\end{equation}
by the tower law for field extensions.  We say $\phi$ is \emph{separable},
\emph{inseparable} or \emph{purely inseparable} according to the extension.  In
particular, when $k$ is of characteristic zero every isogeny $\phi$ is separable.

\begin{prop}
  \label{prop:deg-phi-equals-sizeof-kernel}
  Let $E_{1}$ and $E_{2}$ be elliptic curves over a field $k$ such that there exists
  an isogeny $\phi : E_{1} \rightarrow E_{2}$.  Then $\# \ker{\phi}$ is finite, and
  \begin{equation*}
    deg(\phi) = \# \ker{\phi}.
  \end{equation*}
\end{prop}
\begin{proof}
  See (\cite{silverman86} III, Thm. 4.10).
\end{proof}

An \emph{endomorphism} of an elliptic curve $E$ is an isogeny from $E$ to itself.
The set of all endomorphisms of $E$ forms a ring $End(E)$ under pointwise addition
and composition of morphisms, and is known as the \emph{endomorphism ring} of $E$.

\subsection{Some properties of $End(E)$}
\label{sec:some-properties-ende}

We will show that the endomorphism ring of an elliptic curve has a very particular
structure.  The following example allows us to determine some basic properties.

\begin{example}
  \label{ex:mult-by-m}
  Let $E$ be an elliptic curve given by the equation
  \begin{equation*}
    y^{2} = x^{3} + Ax + B.
  \end{equation*}
  For every rational integer $m$ the
  \emph{multiplication-by-m} map $\left[m\right] \colon E \rightarrow E$ defined by
  % \begin{equation*}
  %   \left[m\right]P = P + \ldots + P \qquad \text{($m$ summands)}
  % \end{equation*}
  \begin{equation*}
    \left[ m \right] P =
    \begin{cases}
      \quad \mathcal{O} & m = 0,\\
      \quad P + \ldots + P & m > 0,\\
      - \left( P + \ldots + P \right) & m < 0,
    \end{cases}
  \end{equation*}
  is an endomorphism of $E$.  Its kernel $E[m]$ is the subgroup of $E$ consisting of
  all points (not just those with co-ordinates in $k$) whose order divides $m$.  From the definition (and the tower law for field extensions) it
  follows that $\left[m\right] \circ \left[n\right] = \left[mn\right]$.
  %%% Commented the following because it's obvious, though may be useful as a space
  %%% filler.
  % Indeed, we have in general that if $P_{1} =
  % \left(x_{1},y_{1}\right)$ and $P_{2} = \left(x_{2},y_{2}\right)$ are points on
  % $E$,
  % then
\end{example}

The multiplication-by-$m$ maps allow us to study $End(E)$, considered as a
$\mathbb{Z}$-module.

\begin{lem}
  \label{lem:mult-by-m-is-non-constant}
  For any non-zero integer $m$, the multiplication-by-$m$ endomorphism is
  non-constant.  In particular, $deg([m]) \geq 1$.
\end{lem}
\begin{proof}
  See (\cite{silverman86} III Prop 4.2).
\end{proof}

\begin{prop}
  \label{prop:End(E)-is-torsion-free}
  Let $E$ be an elliptic curve over a field $k$.  Then the endomorphism ring $End(E)$
  is torsion-free as a $\mathbb{Z}$-module.
\end{prop}
\begin{proof}
  Let $\phi$ be an endomorphism of $E$, and suppose $[m] \circ \phi = [0]$.  Then
  \begin{equation*}
    deg([m])\cdot deg(\phi) = 0,
  \end{equation*}
  so either $[m] = [0]$, or $deg([m]) \geq 1$ by Lemma
  \ref{lem:mult-by-m-is-non-constant} $[m] \neq [0]$, in which case $deg(\phi) = 0$
  whence $\phi = [0]$.
\end{proof}

\begin{cor}
  \label{cor:End(E)-is-char-zero-with-no-zero-divisors}
  The endomorphism ring $End(E)$ of an elliptic curve $E$ is of characteristic zero,
  with no zero divisors.
\end{cor}
\begin{proof}
  Proposition \ref{prop:End(E)-is-torsion-free} above shows that $End(E)$ is of
  characteristic zero, and if $\phi_{1}$ and $\phi_{2}$ are endomorphisms of $E$ such
  that
  \begin{equation*}
    \phi_{1} \circ \phi_{2} = [0],
  \end{equation*}
  then
  \begin{equation*}
    deg(\phi_{1}) \cdot deg(\phi_{2}) = 0,
  \end{equation*}
  whence either $\phi_{1} = [0]$ or $\phi_{2} = [0]$.
\end{proof}

% An elliptic curve $E$ has precisely three points of order 2.  Since $E$ is infinite,
% there exists a point $P_{1}$ on $E$ of order not equal to 2 so that
% $\left[2\right]P_{1} \neq \mathcal{O}$.  Similarly, for a point $P_{2}$ of order 2,
% any odd integer $m$ satisfies $\left[m\right]P_{2} = P_{2}$.  Thus for all non-zero
% $m$ it follows that $\left[m\right] \neq \left[0\right]$.

% As a $\mathbb{Z}$-module, the endomorphism ring of an elliptic curve $E$ is
% torsion-free; if $\phi \in End(E)$ and $m \in \mathbb{Z}$ satisfy
% \begin{equation*}
%   \left[m\right] \circ \phi = \left[ 0 \right]
% \end{equation*}
% then by \eqref{eq:degree-of-composition},
% \begin{equation*}
%   deg \left(\left[m\right]\right) \cdot deg \left( \phi \right) = 0
% \end{equation*}
% whence either $m = 0$ or $deg\left(\left[m\right]\right) > 0$, in which case $\phi =
% \left[0\right]$ and $m \neq 0$.

% Taking degrees also shows that $End(E)$ is an integral domain; if $\phi_{1}$ and
% $\phi_{2}$ are endomorphisms of $E$ such that
% \begin{equation*}
%   \phi_{1} \circ \phi_{2} = \left[0\right],
% \end{equation*}
% then
% \begin{equation*}
%   deg \left( \phi_{1} \right) \cdot deg \left( \phi_{2} \right) = 0,
% \end{equation*}
% from which the result follows.

% We summarise what we have shown in the following proposition.

% \begin{prop}
%   \label{prop:End(E)-is-char-zero-id}
%   Let $E$ be an elliptic curve.  Then $End(E)$ is a characteristic zero integral
%   domain.
% \end{prop}

An elliptic curve whose endomorphism ring is strictly larger than $\mathbb{Z}$ is
said to have \emph{complex multiplication}.  We will see shortly (Proposition \ref{prop:properties-of-m-dual}) that $deg([m]) =
m^{2}$.  Given that this is true, it is clear that an elliptic curve $E$ has complex
multiplication if and only if it possesses an endomorphism $\phi$ whose degree is a non-square.

\begin{example}
  \label{ex:cm-example}
  Consider the elliptic curve $E$ given by the equation
  \begin{equation*}
    y^{2} = x^{3} + x.
  \end{equation*}
  The map $\left[ i \right] \colon E \rightarrow E$ given by
  \begin{equation*}
    \left[ i \right](x,y) = (-x,iy)
  \end{equation*}
  is an endomorphism of $E$.  Note that $\left[ i \right]^{2} = \left[ -1 \right]$,
  so that $\left[ i \right] \neq \left[ m \right]$ for any rational integer $m$.
  Thus $E$ has complex multiplication.
  % Worked example showing End(E) = Z[i]
\end{example}

\begin{example}
  \label{ex:finite-field-cm}
  Let $k$ be a field of characteristic $q = p^{r}$, and let $E$ be an elliptic curve
  over $k$.  We define $E^{(q)}$ to be the curve obtained by raising the coefficients
  of the Weierstrass equation of $E$ to the $q$-th power.  The \emph{Frobenius
    morphism} $\phi_{q} : E \rightarrow E^{(q)}$ is given by
  \begin{equation*}
    \phi_{q}(x,y) = (x^{q},y^{q}).
  \end{equation*}
  Now, if $k = \mathbb{F}_{q}$ is a finite field, then $k^{*}$ is cyclic of order $q
  - 1$ so that $E^{(q)} = E$, and $\phi_{q}$ is actually an endomorphism of $E$.  It
  is known (\cite{silverman86} II Prop 2.1) that $deg(\phi_{q}) = q$, so that when
  $q$ is a non-square (i.e. $r$ is odd), then every elliptic curve defined over $k$
  has complex multiplication.
\end{example}

\begin{prop}
  \label{prop:iso-curves-iso-end-rings}
  Let $E_{1}$ and $E_{2}$ be isomorphic elliptic curves.  Then $End(E_{1})$ is
  isomorphic to $End(E_{2})$.
\end{prop}
\begin{proof}
  Let $f \colon E_{1} \rightarrow E_{2}$ denote the isomorphism, and let $\phi$ be an
  endomorphism of $E_{1}$. We determine an isomorphism $F \colon End(E_{1}) \rightarrow
  End(E_{2})$ by the following commutative diagram:
\begin{equation*}
\xymatrix{
  E_{1} \ar[d]^{f} \ar[r]^{\phi} & E_{1} \ar[d]^{f} \\
  E_{2} \ar[r] & E_{2}}
\end{equation*}
i.e. $F(\phi) = f \circ \phi \circ f^{-1}$.
\end{proof}
We will see shortly that the endomorphism ring of any elliptic curve with complex
multiplication (in characteristic zero) has the structure of an order in an imaginary
quadratic field.  In the next two sections we develop the technical tools required to
prove this result.

\subsection{Dual isogenies}
\label{sec:an-interlude-dual}

Let $E_{1}$ and $E_{2}$ be elliptic curves.  For every non-zero isogeny $\phi \colon
E_{1} \rightarrow E_{2}$ there exists a unique isogeny $\hat{\phi} \colon E_{2}
\rightarrow E_{1}$ which satisfies
\begin{equation}
  \label{eq:dual-isogeny}
  \hat{\phi} \circ \phi = \left[ m \right],
\end{equation}
where $m$ is the degree of $\phi$ (when $\phi = \left[ 0 \right]$ we define
$\hat{\phi}$ to be $\left[ 0 \right]$).  We say $\hat{\phi}$ is the \emph{dual
  isogeny} to $\phi$.  Note that, for any elliptic curve $E$, we have $\widehat{[1]}
= [1]$.  This follows from the definition of the dual isogeny and the ring structure
of $End(E)$.

The dual isogeny will be a useful tool in studying the multiplication-by-$m$ maps.
Some basic properties are given in the following lemma:
\begin{lem}
  \label{lem:properties-of-dual-isogenies}
  Let $\phi \colon E_{1} \rightarrow E_{2}$ be an isogeny of degree $d$.  Then
  \begin{enumerate}[(i)]
  \item $\bar{\phi} \circ \phi = \left[ d \right]$ on $E_{1}$, and $\phi \circ
    \bar{\phi} = \left[ d \right]$ on $E_{2}$,
  \item if $\theta \colon E_{2} \rightarrow E_{3}$ is another isogeny, then
    \begin{equation*}
      \widehat{\theta \circ \phi} = \hat{\phi} \circ \hat{\theta},
    \end{equation*}
  \item if $\psi \colon E_{1} \rightarrow E_{2}$ is another isogeny, then
    \begin{equation*}
      \widehat{\phi + \psi} = \hat{\phi} + \hat{\psi}.
    \end{equation*}
  \end{enumerate}
\end{lem}
\begin{proof}
  See (\cite{silverman86} III \S 6, Thm 6.2).
\end{proof}

\begin{prop}
  \label{prop:properties-of-m-dual}
  Let $E$ be an elliptic curve over a field $k$, and let $m$ be a non-zero integer.
  Then
  \begin{enumerate}[(i)]
  \item $\widehat{[m]} = [m]$,
  \item $deg([m]) = m^{2}$.
  \end{enumerate}
\end{prop}

\begin{proof}
  To prove (i) we proceed by induction.  It is obvious that $\widehat{[0]} = [0]$ and
  $\widehat{[1]} = [1]$.  Now let $m$ be an arbitrary integer (for simplicity,
  suppose $m$ is positive; the proof is hardly changed otherwise). Then
  \begin{eqnarray*}
    \widehat{[m]}&=&\widehat{[(m-1) + 1]}\\
    &=&\widehat{[m-1]} + \widehat{[1]} \quad \text{(by Lemma
      \ref{lem:properties-of-dual-isogenies} (i)) }\\
    &=&[m-1] + [1] \quad \text{(by the inductive hypothesis)}\\
    &=&[m],
  \end{eqnarray*}
  thus completing the proof of (i).

  Now let $d = deg([m])$.  Then, by Lemma \ref{lem:properties-of-dual-isogenies} (i)
  we have
  \begin{equation*}
    [d] = [m] \circ \widehat{([m])} = [m] \circ [m] = [m^{2}],
  \end{equation*}
  and since $End(E)$ is torsion-free we must have $d = m^{2}$ as required.
\end{proof}

\begin{cor}
  \label{cor:structure-of-E[m]}
  Let $E$ and $m$ be as above.  Then
  \begin{equation*}
    E[m] \cong \mathbb{Z} / m \mathbb{Z} \times \mathbb{Z} / m \mathbb{Z}.
  \end{equation*}
\end{cor}
\begin{proof}
  We proceed by induction.  For $m = 1$ and $m = 2$ the result is clear.  So suppose $m > 2$.  By
  Proposition \ref{prop:deg-phi-equals-sizeof-kernel} we have
  \begin{equation*}
    \# E[d] = deg([d]) = d^{2},
  \end{equation*}
  for every integer $d$ dividing $m$.  Now recall that a finite abelian group $G$ of
  order $g$ is isomorphic to the direct product of its $p$-primary components
  $G_{p}$, where
  \begin{equation*}
    G_{p} = \{ s \in G : s \text{ has } p \text{-power order} \},
  \end{equation*}
  for each prime $p$ dividing $g$.  If $G = E[m]$ then its $p$-primary components are
  precisely the subgroups $E[p_{i}^{e_{i}}]$, where $m = p_{1}^{e_{1}} \ldots
  p_{r}^{e_{r}}$ is the prime factorisation of $m$.  In particular, each of the
  $E[p_{i}^{e_{i}}]$ is (by the inductive hypothesis) isomorphic to $\mathbb{Z} /
  p_{i}^{e_{i}} \mathbb{Z} \times \mathbb{Z} / p_{i}^{e_{i}} \mathbb{Z}$.  This
  completes the proof.
\end{proof}

\subsection{The Tate module}
\label{sec:tate-module}

Let $E$ be an elliptic curve over a field $k$ of characteristic zero, and let $\ell$
be a rational prime.  For every positive integer $n$ the multiplication-by-$\ell$ map
takes $E[\ell^{n+1}]$ into $E[\ell^{n}]$.  We thus define the \emph{$\ell$-adic Tate
  module of $E$} to be the projective limit
\begin{equation*}
  T_{\ell}(E) =  \lim_{\overleftarrow{n}} E[\mathcal{\ell}^{n}].
\end{equation*}
As a $\mathbb{Z}$-module $E[\ell^{n}]$ is clearly annihilated by $\ell^{n}$, and
hence by the ideal $\ell^{n}\mathbb{Z}$, so that each of the $E[\ell^{n}]$ has the
structure of a $\mathbb{Z}/\ell^{n}\mathbb{Z}$-module.  Then $T_{\ell}(E)$, being a
projective limit of $\mathbb{Z}/\ell^{n}\mathbb{Z}$-modules, has the structure of a
$\mathbb{Z}_{\ell}$-module.  Furthermore, since $E[\ell^{n}] \cong
\mathbb{Z}/\ell^{n}\mathbb{Z} \times \mathbb{Z}/\ell^{n}\mathbb{Z}$ by Corollary
\ref{cor:structure-of-E[m]}, it follows immediately from the definition that
\begin{equation}
  \label{eq:structure-of-Tate-module}
  T_{\ell}(E) \cong \mathbb{Z}_{\ell} \times \mathbb{Z}_{\ell}.
\end{equation}

Let $\phi$ be an endomorphism of $E$.  Note that if $P$ is a point in $E[m]$ then
\begin{eqnarray*}
  [m] \circ \phi (P)&=&\phi (P) + \ldots + \phi (P)\\
  &=&\phi ([m] P) \quad \text{(since $\phi$ is a homomorphism)}\\
  &=&\phi (\mathcal{O}) \quad \text{(since $P \in E[m]$)}\\
  &=&\mathcal{O} \quad \text{(by definition of $\phi$)}
\end{eqnarray*}
so that $\phi (E[m]) \subset E[m]$ for all $m$.  In particular, taking $m$ =
$\ell^{n}$ for varying $n$ shows that every endomorphism $\phi$ of $E$ induces an
endomorphism $\phi_{\ell}$ of $T_{\ell}(E)$, i.e. there is a homomorphism $End(E)
\rightarrow T_{\ell}(E)$ given by
\begin{equation}
  \label{eq:End(E)-to-T_ell(E)}
  \phi \rightarrow \phi_{\ell},
\end{equation}
and extending \eqref{eq:End(E)-to-T_ell(E)} to $End(E) \otimes \mathbb{Z}_{\ell}$ gives the following
result.

\begin{lem}
  \label{lem:End(E)-tensor-Z_ell-injects}
  The homomorphism $End(E) \otimes \mathbb{Z}_{\ell} \rightarrow End(T_{\ell}(E))$
  induced by \eqref{eq:End(E)-to-T_ell(E)} is injective.
\end{lem}
\begin{proof}
  See (\cite{silverman86} III Thm. 7.7).
\end{proof}

It follows that $End(E) \otimes \mathbb{Z}_{\ell}$ has $\mathbb{Z}_{\ell}$-rank at
most $4$, since $End(T_{\ell}(E)) \cong M_{2}(\mathbb{Z}_{\ell})$ by
\eqref{eq:structure-of-Tate-module}.

\begin{prop}
  \label{prop:rank-of-End(E)}
  The endomorphism ring $End(E)$ of an elliptic curve $E$ is a characteristic zero
  integral domain of at most rank $4$ over $\mathbb{Z}$.
\end{prop}
\begin{proof}
  Only the last statement needs proving.  We have
  \begin{equation*}
    rank_{\mathbb{Z}}(End(E)) = rank_{\mathbb{Z}_{\ell}}(End(E) \otimes \mathbb{Z}_{\ell}),
  \end{equation*}
  since if $(x_{i})$ is a set of basis elements of $End(E)$ over $\mathbb{Z}$ then
  $(x_{i} \otimes 1)$ is a $\mathbb{Z}_{\ell}$-basis for $End(E) \otimes
  \mathbb{Z}_{\ell}$.  So $rank_{\mathbb{Z}}(End(E)) \leq 4$ as required.
\end{proof}

%%% Stuff about action of Gal on Tate module being abelian.  Put this in when talking
%%% about the maximal abelian extension.
% For every integer $m$ there is a natural action of the Galois group
% $Gal_{\bar{k}/k}$ on $E[m]$ given by $\sigma \cdot P = P^{\sigma}$, where if $P$ is
% given by homogeneous co-ordinates $[x : y : z]$ then $P^{\sigma} = [x^{\sigma} :
% y^{\sigma} : z^{\sigma}]$.  This action as well-defined, for
% \begin{eqnarray*}
%   [m](P^{\sigma}) & = & (P^{\sigma} + \ldots + P^{\sigma})\\
%   & = & (P + \ldots + P)^{\sigma}\\
%   & = & ([m]P)^{\sigma},
% \end{eqnarray*}
% so that if $[m]P = \mathcal{O}$ then $[m](P^{\sigma}) = \mathcal{O}^{\sigma} =
% \mathcal{O}$.

% \begin{equation*}
% \xymatrix{
%   E[\ell^{n+1}] \ar[d]^{\sigma} \ar[r]^{[\ell]} & E[\ell^{n}] \ar[d]^{\sigma} \\
%   E[\ell^{n+1}] \ar[r]^{[\ell]} & E[\ell^{n}]}
% \end{equation*}

\subsection{The action of $Gal(\bar{k} / k)$ on $E[m]$}
\label{sec:action-of-Galois-on-m-torsion}

Let $E$ be an elliptic curve defined over some field $k$.  Let $P = (x,y)$ be a point
on $E$ and let $\sigma$ be an element of $Gal(\bar{k} / k)$.  We define $P^{\sigma}$
to be the point $P^{\sigma} = (x^{\sigma},y^{\sigma})$.  If
$P$ is in $E[m]$ for some $m$, then
\begin{equation*}
  [m](P^{\sigma}) = (f_{1}(P^{\sigma}),f_{2} (P^{\sigma}))
\end{equation*}
where $f_{1}$ and $f_{2}$ are the rational functions which define $[m]$.  But each
$f_{i}$ can be viewed as a quotient of polynomials in $k[x,y] / (y^{2} - (x^{3} + Ax
+ B)$.  Then, since $A$ and $B$ are in $k$, and $\sigma$ fixes $k$, we have
$f_{i}^{\sigma}$ = $f_{i}$.  Hence,
\begin{equation*}
  [m](P^{\sigma}) = (f_{1}^{\sigma}(P^{\sigma}),f_{2}^{\sigma}(P^{\sigma})) =
  ([m]P)^{\sigma} = \mathcal{O}^{\sigma} = \mathcal{O}
\end{equation*}

\begin{lem}
  \label{lem:E[m]-is-stable-under-Galois}
  Let $E$ be an elliptic curve over $k$.  There is a well-defined action of
  $Gal(\bar{k} / k)$ on $E[m]$, given by
  \begin{equation*}
    P^{\sigma} = P, \quad P \in E[m], \sigma \in Gal(\bar{k} / k).
  \end{equation*}
\end{lem}
\begin{proof}
  This is clear by the above remarks.
\end{proof}

A consequence of Lemma \ref{lem:E[m]-is-stable-under-Galois} is that we have a
representation
\begin{equation*}
\rho : Gal(\bar{k} / k) \rightarrow Aut(E[m]),
\end{equation*}
given by
\begin{equation}
  \label{eq:Galois-representation}
  \rho (\sigma) (P) = P^{\sigma}.
\end{equation}
Note also, that by Corollary \ref{cor:structure-of-E[m]}, we have that $Aut(E[m])
\cong GL_{2}(\mathbb{Z} / m \mathbb{Z})$.

\subsection{The structure of $End(E)$ in characteristic zero}
\label{sec:struct-ende-char-zero}

We are now in a position to describe the general structure of the endomorphism ring
of an elliptic curve.

\begin{thm}
  \label{thm:structure-thm-for-End(E)}
  Let $E$ be an elliptic curve over a field $k$ of characteristic zero.  Then either
  $End(E) \cong \mathbb{Z}$ or $End(E)$ is isomorphic to an order in a quadratic
  imaginary field.
\end{thm}
\begin{proof}
  Let $K = End(E) \otimes \mathbb{Q}$.  For each $\alpha \in \mathbb{Q}, \phi \in
  End(E)$ we define an extended dual $\widehat{\alpha \cdot \phi}$ by
  \begin{equation*}
    \widehat{\alpha \cdot \phi} = \alpha \cdot \widehat{\phi},
  \end{equation*}
  where $\widehat{\phi}$ is the dual isogeny to $\phi$.  We define functions $N : K
  \rightarrow \mathbb{Q}$ and $T : K \rightarrow \mathbb{Q}$ by
  \begin{equation*}
    N \Phi = \Phi \cdot \widehat{\Phi} \quad \text{and} \quad T \Phi = \Phi + \widehat{\Phi}.
  \end{equation*}
  Note that, by Proposition \ref{prop:properties-of-m-dual}, the value of $N \Phi$ is
  the product of a positive rational number and a positive integer, so that $N \Phi$
  is a positive rational number.  Furthermore, we have
  \begin{equation*}
    N(\Phi - 1) = (\Phi - 1) \cdot \widehat{(\Phi - 1)} = N\Phi - T\Phi + 1,
  \end{equation*}
  so that
  \begin{equation*}
    T\Phi = 1 + N\Phi - N(\Phi - 1)
  \end{equation*}
  is indeed rational.  It is clear from the definitions that $T$ is
  $\mathbb{Q}$-linear, and that $T\alpha = 2\alpha$ for $\alpha$ in $\mathbb{Q}$
  (here of course we identify $\alpha \in \mathbb{Q}$ with $\alpha \cdot [1] \in K$).
  Now, if $K = \mathbb{Q}$ we are done.  Otherwise, choose some $\Phi$ in $K -
  \mathbb{Q}$.  We may assume, without loss of generality, that $T\Phi = 0$, for we
  are free to replace $\Phi$ with $\Phi - \frac{1}{2}T\Phi \in K - \mathbb{Q}$ (since
  $\frac{1}{2}T\Phi \in \mathbb{Q}$), and
  \begin{eqnarray*}
    T(\Phi - \frac{1}{2}T\Phi)&=&T \Phi - \frac{1}{2} T (T\Phi)\\
    &=&T \Phi - \frac{1}{2}(2T \Phi)\\
    &=&0.
  \end{eqnarray*}
  Then, we have
  \begin{eqnarray*}
    0&=&(\Phi - \Phi)(\Phi - \widehat{\Phi})\\
    &=&\Phi^{2} - (T\Phi)\Phi + N\Phi\\
    &=&\Phi^{2} + N\Phi,
  \end{eqnarray*}
  so that $\Phi^{2}$ is a negative rational number.  Hence, $\mathbb{Q}(\Phi)$ is a
  quadratic imaginary field.
  Now, by Proposition \ref{prop:rank-of-End(E)}, it is possible that $K$ is a
  $4$-dimensional $\mathbb{Q}$-vector space.  However, this can only happen when $k$
  is of prime characteristic (in this case $K$ could be a quaternion algebra).  See
  (\cite{silverman86} VI Thm. 6.16) for a complex analytic proof of this.  Since we
  assume that $k$ is of characteristic zero, we must have $K = \mathbb{Q}(\Phi)$,
  which completes the proof.
\end{proof}

In Example \ref{ex:cm-example} we saw that the endomorphism ring of the elliptic
curve $E$ with equation $y^{2} = x^{3} + x$ contains the ring $\mathbb{Z}[i]$ of
Gaussian integers.  It follows by Theorem \ref{thm:structure-thm-for-End(E)} that the
endomorphism ring of $E$ is precisely $\mathbb{Z}[i]$.

%%% Local Variables:
%%% mode: latex
%%% TeX-master: "../main"
%%% End:
