\message{ !name(../main.tex)}
\documentclass{amsart}
% \documentclass{article}
\usepackage{amsfonts,amsmath,amsthm,enumerate,mathrsfs}
\usepackage[all,cmtip]{xy}


% amsthm stuff
\newtheorem{thm}{Theorem}[subsection]
\newtheorem{prop}{Proposition}[subsection]
\newtheorem{lem}{Lemma}[subsection]
\newtheorem{cor}{Corollary}[subsection]

\theoremstyle{definition}
\newtheorem{example}{Example}[subsection]
\newtheorem{rem}{Remark}[subsection]

\title{Elliptic Curves with Complex Multiplication}

\begin{document}

\message{ !name(cm_rational/rational.tex) !offset(-21) }
\section{Complex multiplication over algebraic extensions of $\mathbb{Q}$}
\label{sec:compl-mult-over-Q}

TODO: intro.

Let $E$ be an elliptic curve defined over $\mathbb{C}$, given by the equation
\begin{equation*}
  y^{2} = x^{3} + Ax + B,
\end{equation*}
and let $\sigma$ be a field automorphism of $\mathbb{C}$. We define $E^{\sigma}$ to be
the elliptic curve obtained by letting $\sigma$ act on the coefficients of the equation
of $E$, i.e. $E^{\sigma}$ is obtained from the equation
\begin{equation*}
  y^{2} = x^{3} + A^{\sigma}x + B^{\sigma}.
\end{equation*}
It is clear that $E^{\sigma} = \{ P^{\sigma} : P \in E \}$, for if $P = (x,y)$, then
$P^{\sigma} = (x^{\sigma},y^{\sigma})$ satisfies
\begin{equation*}
  (y^{\sigma})^{2} - ( (x^{\sigma})^{3} + A^{\sigma}(x^{\sigma}) + B^{\sigma} ) = (y^{2} -
  (x^{3} + Ax + B))^{\sigma} = 0^{\sigma} = 0.
\end{equation*}
Now, if $\phi$ is an endomorphism of $E$, then $\phi$ is given by rational maps in
the function field $\mathbb{C}(E)$:
\begin{equation*}
  \phi = (f_{1},f_{2}), \quad f_{i} \in \mathbb{C}(E),
\end{equation*}
and we define an endomorphism $\phi^{\sigma}$ of $E^{\sigma}$ in the obvious way:
\begin{equation*}
  \phi^{\sigma} = (f_{1}^{\sigma},f_{2}^{\sigma}).
\end{equation*}
Note that, if $(x\prime{},y\prime{})$

\begin{prop}
  \label{prop:j(E)-is-in-Q}
  Let $E$ be an elliptic curve
\end{prop}
%%% Local Variables: 
%%% mode: latex
%%% TeX-master: "../main"
%%% End: 

\message{ !name(../main.tex) !offset(-30) }

\end{document}

%%% Local Variables: 
%%% mode: latex
%%% TeX-master: t
%%% End: 
