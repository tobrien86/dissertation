\message{ !name(../main.tex)}
% \documentclass{amsart}
\documentclass{article}
\usepackage{amsfonts,amsmath,amsthm,enumerate,mathrsfs}
\usepackage[all,cmtip]{xy}


% amsthm stuff
\newtheorem{thm}{Theorem}[subsection]
\newtheorem{prop}{Proposition}[subsection]
\newtheorem{lem}{Lemma}[subsection]
\newtheorem{cor}{Corollary}[subsection]

\theoremstyle{definition}
\newtheorem{example}{Example}[subsection]
\newtheorem{rem}{Remark}[subsection]

\title{Elliptic Curves with Complex Multiplication}

\begin{document}

\message{ !name(endomorphisms/end.tex) !offset(432) }
\subsection{The structure of $End(E)$ in characteristic zero}
\label{sec:struct-ende-char-zero}

TODO: Short introduction.

\begin{thm}
  \label{thm:structure-thm-for-End(E)}
  Let $E$ be an elliptic curve over a field $k$ of characteristic zero.  Then either
  $End(E) \cong \mathbb{Z}$ or $End(E)$ is isomorphic to an order in a quadratic
  imaginary field.
\end{thm}
\begin{proof}
  Let $K = End(E) \otimes \mathbb{Q}$.  For each $\alpha \in \mathbb{Q}, \phi \in
  End(E)$ we define an extended dual $\widehat{\alpha \cdot \phi}$ by
  \begin{equation*}
    \widehat{\alpha \cdot \phi} = \alpha \cdot \widehat{\phi},
  \end{equation*}
  where $\widehat{\phi}$ is the dual isogeny to $\phi$.  We define functions $N : K
  \rightarrow \mathbb{Q}$ and $T : K \rightarrow \mathbb{Q}$ by
  \begin{equation*}
    N \Phi = \Phi \widehat{\Phi}
  \end{equation*}
\end{proof}

In Example \ref{ex:cm-example} we saw that the endomorphism ring of the elliptic
curve $E$ with equation $y^{2} = x^{3} + x$ contains the ring $\mathbb{Z}[i]$ of
Gaussian integers.  It follows by Theorem \ref{thm:structure-thm-for-End(E)} that the
endomorphism ring of $E$ is precisely $\mathbb{Z}[i]$.

%%% Local Variables: 
%%% mode: latex
%%% TeX-master: "../main"
%%% End: 

\message{ !name(../main.tex) !offset(-18) }

\end{document}

%%% Local Variables: 
%%% mode: latex
%%% TeX-master: t
%%% End: 
