\newpage
\section{Elliptic curves over $\mathbb{C}$}
\label{sec:compl-mult-over-C}

Our ultimate aim is to develop the theory of complex multiplication for elliptic
curves defined over $\mathbb{\bar{Q}}$.  In Section \ref{sec:towards-abel-extens} we
prove that any elliptic curve defined over $\mathbb{C}$ with complex
multiplication is isomorphic to an elliptic curve defined over $\mathbb{\bar{Q}}$.
We turn our attention thus to the complex theory; the main benefit of which (from our
point of view) is that an isogeny of complex elliptic curves has a very simple
geometric interpretation.

\subsection{A brief review of elliptic curves over $\mathbb{C}$}
\label{sec:review-complex-curves}



The material relating to the lattices and functions considered in this section can be
found in (\cite{serre73} VII).

Recall that a \emph{lattice} in $\mathbb{C}$ is a subgroup $\Lambda$ of $\mathbb{C}$ of
$\mathbb{Z}$-rank $2$, with a $\mathbb{Z}$-basis $(\omega_{1},\omega_{2})$ which
spans $\mathbb{C}$ over $\mathbb{R}$.  The following lemma will not be used until
later, but is convenient to prove here.

\begin{lem}
  \label{lem:latice-iff-discrete}
  A subgroup $\Lambda$ of $\mathbb{C}$ is a lattice if and only if it is a discrete
  subgroup of $\mathbb{C}$, which spans $\mathbb{C}$ over $\mathbb{R}$.
\end{lem}
\begin{proof}
  Todo: This.
\end{proof}

Let $\Lambda$ be a lattice in $\mathbb{C}$.  Recall the Weierstrass $\wp$-function:
\begin{equation*}
  \wp(z,\Lambda)\footnote{We will usually supress the $\Lambda$ and simply write $\wp(z)$} = \frac{1}{z^{2}} + \sum_{\omega \in \Lambda - 0}
  \frac{1}{(z-\omega)^{2}} - \frac{1}{\omega^{2}},
\end{equation*}
and its derivative
\begin{equation*}
  \wp^{\prime}(z,\Lambda) = -2 \sum_{\omega \in \Lambda}\frac{1}{(z - \omega)^{3}}.
\end{equation*}
The $\wp$-function is doubly-periodic, and thus descends to a well-defined function
on the torus $\mathbb{C} / \Lambda$.  Recall also the Eisenstein series for $\Lambda$
of weight $2k$:
\begin{equation*}
  G_{2k}(\Lambda) = \sum_{\omega \in \Lambda - 0} \frac{1}{\omega^{2k}}.
\end{equation*}
There is a relation of algebraic dependence between $\wp$ and its derivative, given
by:
\begin{equation*}
  \wp'(z)^{2} = 4\wp(z)^{3} - g_{2}\wp(z) - g_{3},
\end{equation*}
where $g_{2} = 60G_{4}(\Lambda)$ and $g_{3} = 140G_{6}(\Lambda)$.  So if
$E_{\Lambda}$ is the curve in $\mathbb{P}^{2}_{\mathbb{C}}$ defined by the equation
\begin{equation}
  \label{eq:complex-eqn-for-ec}
  y^{2} = 4x^{3} - g_{2}x - g_{3},
\end{equation}
then there is a holomorphic bijection of Riemann surfaces $\mathbb{C}/\Lambda
\rightarrow E_{\Lambda}$ given by
\begin{equation*}
  z + \Lambda \rightarrow
  \begin{cases}
    [\wp(z) : \wp'(z) : 1] & z \neq 0,\\
    [0 : 1 : 0] & z = 0.
  \end{cases}
\end{equation*}
The curve defined in \eqref{eq:complex-eqn-for-ec} is non-singular provided the
discriminant
\begin{equation*}
  \Delta(E_{\Lambda}) = g_{2}^{3} - 27g_{3}^{2}
\end{equation*}
is non-zero.

\subsection{Endomorphisms}
\label{sec:endomorphisms}

One advantage of working over the complex numbers is that the endomorphism ring of an
elliptic curve can be interpreted in a simple way in terms of the lattice which
defines the curve.

Let $\Lambda_{1}$ and $\Lambda_{2}$ be lattices in $\mathbb{C}$. Suppose $\alpha \in
\mathbb{C}$ is such that $\alpha\Lambda_{1} \subset \Lambda_{2}$.  Then $\alpha$
defines a map $\phi_{\alpha} \colon \mathbb{C}/\Lambda_{1} \rightarrow
\mathbb{C}/\Lambda_{2}$ given by
\begin{equation}
  \label{eq:phi_alpha}
  \phi_{\alpha}(z + \Lambda_{1}) = \alpha z + \Lambda_{2}.
\end{equation}
This map is well defined; if $\omega \in \Lambda_{1}$, then
\begin{equation*}
  \alpha(z + \omega) = \alpha z + \alpha\omega \equiv \alpha z \pmod{\Lambda_{2}}.
\end{equation*}
The maps $\phi_{\alpha}$ are clearly holomorphic group homomorphisms.

\begin{prop}
  \label{prop:complex-isogenies}
  Let $\Lambda_{1}$ and $\Lambda_{2}$ be lattices in $\mathbb{C}$ and let
  $E_{\Lambda_{1}}$ and $E_{\Lambda_{2}}$ be the corresponding elliptic curves given
  by \eqref{eq:complex-eqn-for-ec}.  Then $Hom(E_{\Lambda_{1}},E_{\Lambda_{2}})$ is
  isomorphic to $\{\alpha \in \mathbb{C} \colon \alpha\Lambda_{1} \subset
  \Lambda_{2}\}$.
\end{prop}
\begin{proof}
  We give a sketch of the proof, describing the maps involved. See
  (\cite{silverman86} VI, Thm 4.1) for the full proof.

  Let $Holom(\Lambda_{1},\Lambda_{2})$ denote the set of holomorphic maps $\phi :
  \mathbb{C} / \Lambda_{1} \rightarrow \mathbb{C} / \Lambda_{2}$ with $\phi (0) = 0$.
  The idea is to show that there are bijections between the following two pairs of
  sets:
  \begin{enumerate}[(i)]
  \item $\{ \alpha \in \mathbb{C} : \alpha \Lambda_{1} \subset \Lambda_{2} \}
    \rightarrow Holom(\Lambda_{1},\Lambda_{2})$,
  \item $\{ \text{isogenies } E_{\Lambda_{1}} \rightarrow E_{\Lambda_{2}} \}
    \rightarrow Holom(\Lambda_{1},\Lambda_{2})$,

   \end{enumerate}
   where the map in (i) is given by $\alpha \rightarrow \phi_{a}$ (where
   $\phi_{\alpha}$ is given by \eqref{eq:phi_alpha}), and the map in (ii) is natural
   inclusion.

   We have already seen that $\phi_{\alpha}$ is in $Holom(\Lambda_{1},\Lambda_{2})$,
   so the mapping in (i) is well defined. The discreteness of $\Lambda_{2}$ shows
   that the mapping is injective. Conversely, any $\phi$ in
   $Holom(\Lambda_{1},\Lambda_{2})$ lifts to a holomorphic map $\Phi : \mathbb{C}
   \rightarrow \mathbb{C}$ with $\Phi (0) = 0$ such that the following diagram
   commutes:
   \begin{equation*}
     \xymatrix{
       \mathbb{C} \ar[d] \ar[r]^{\Phi} & \mathbb{C}  \ar[d]\\
       \mathbb{C} / \Lambda_{1} \ar[r]^{\phi }& \mathbb{C} / \Lambda_{2}
     }.
   \end{equation*}
   It turns out that $\Phi (\Lambda_{1}) \subset \Lambda_{2}$ and that the derivative
   $\Phi^{\prime}$ of $\Phi$ is a holomorphic elliptic function and is thus constant.
   Thus $\Phi (z) = \alpha z + \beta$, and $\Phi (0) = 0$ shows that $\beta = 0$.
   Therefore $\phi (z) = \alpha z$ for some non-zero $\alpha$ such that $\alpha
   \Lambda_{1} \subset \Lambda_{2}$, so $\phi = \phi_{\alpha}$, showing the mapping
   in (i) is surjective.

   Now consider the mapping in (ii).  Since an isogeny $\phi : E_{\Lambda_{1}}
   \rightarrow E_{\Lambda_{2}}$ is everywhere locally defined, it defines unique a
   holomorphic map of Riemann surfaces $\mathbb{C} / \Lambda_{1} \rightarrow
   \mathbb{C} / \Lambda_{2}$.  Conversely any holomorphic map $\phi : \mathbb{C} /
   \Lambda_{1} \rightarrow \mathbb{C} / \Lambda_{2}$ is given by $\phi_{\alpha}$ for
   some non-zero $\alpha$ satisfying $\alpha \Lambda_{1} \subset \Lambda_{2}$, and so
   defines a map $E_{\Lambda_{1}} \rightarrow E_{\Lambda_{2}}$ by
   \begin{equation*}
     (\wp (z,\Lambda_{1}), \wp^{\prime}(z,\Lambda_{1})) \rightarrow (\wp (\alpha z,
     \Lambda_{2}), \wp^{\prime} (\alpha z , \Lambda_{2})).
   \end{equation*}
   Since $\alpha \Lambda_{1} \subset \Lambda_{2}$ it follows that both $\wp (\alpha
   z, \Lambda_{2})$ and its derivative are both elliptic functions of $\Lambda_{1}$,
   and every such function can be expressed as a rational function in $\wp (z,
   \Lambda_{1})$ and $\wp^{\prime} (z, \Lambda_{1})$, which shows that $\phi$ is an
   isogeny.
 \end{proof}

 Recall that we say two lattices $\Lambda_{1}$ and $\Lambda_{2}$ are
 \emph{homothetic} if there exists a non-zero $\alpha$ in $\mathbb{C}$ such that
 $\Lambda_{2} = \alpha \Lambda_{1}$.

 \begin{cor}
   \label{cor:homothetic-lattices-give-isomorphic-curves}
   Let $\Lambda_{1}$ and $\Lambda_{2}$ be as above.  Then the curves
   $E_{\Lambda_{1}}$ and $E_{\Lambda_{2}}$ are isomorphic if and only if
   $\Lambda_{1}$ is homothetic to $\Lambda_{2}$.
 \end{cor}

 \begin{proof}
   If $E_{1} \cong E_{2}$ then there exist isogenies $\phi_{1} : E_{1} \rightarrow
   E_{2}$ and $\phi_{2} : E_{2} \rightarrow E_{1}$ which satisfy
   \begin{equation*}
     \phi_{2} \circ \phi_{1} = id_{E_{1}} \quad \text{and} \quad \phi_{1} \circ
     \phi_{2} = id_{E_{2}}.
   \end{equation*}
   Let $\alpha$ and $\beta$ be non-zero elements of $\mathbb{C}$ such that $\alpha
   \Lambda_{1} \subset \Lambda_{2}$ and $\beta \Lambda_{2} \subset \lambda_{1}$, and
   $\phi_{1}$ and $\phi_{2}$ correspond (as in the proof of Proposition
   \ref{prop:complex-isogenies}) to the holomorphic functions $\phi_{\alpha}$ and
   $\phi_{\beta}$ respectively.  Since $\phi_{1}$ and $\phi_{2}$ compose to the
   identity we must have $\beta = \alpha^{-1}$, so that
   \begin{equation*}
     \alpha \Lambda_{1} \subset \Lambda_{2} \subset \alpha \Lambda_{1}.
   \end{equation*}
   So $\Lambda_{2} = \alpha \Lambda_{1}$ as required.  Conversely, if $\Lambda_{1}$
   and $\Lambda_{2}$ are homothetic then $\Lambda_{2} = \alpha \Lambda_{1}$ for some
   non-zero $\alpha$.  Then the map $\phi_{\alpha} : \mathbb{C} / \Lambda_{1}
   \rightarrow \mathbb{C} / \alpha \Lambda_{1}$ is clearly a bijection with a
   holomorphic inverse, so $E_{\Lambda_{1}}$ and $E_{\Lambda_{2}}$ are isomorphic by
   the commutativity of the following diagram:
   \begin{equation*}
     \xymatrix{
       \mathbb{C} / \Lambda_{1} \ar[d] \ar[r]^{\Phi} & \mathbb{C} / \alpha \Lambda_{1} \ar[d]\\
       E_{\Lambda_{1}} \ar[r]^{\phi }& E_{\Lambda_{2}}
     },
   \end{equation*}
   thus completing the proof.
 \end{proof}

 The next result is known as the \emph{Uniformisation theorem}.  It is useful in that
 it parametrises every elliptic curve over $\mathbb{C}$

 \begin{thm}
   \label{thm:uniformisation-theorem}
   Let $A$ and $B$ be complex numbers which satisfy
   \begin{equation*}
     A^{3} - 27B^{2} \neq 0.
   \end{equation*}
   Then there exists a unique lattice $\Lambda \subset \mathbb{C}$ such that
   \begin{equation*}
     g_{2}(\Lambda) = A \quad \text{and} \quad g_{3}(\Lambda) = B.
   \end{equation*}
 \end{thm}
 \begin{proof}
   TODO: Reference Shimura.
 \end{proof}

 An obvious consequence of the Uniformisation is the following corollary:

 \begin{cor}
   \label{cor:uniformisation-corollary}
   Let $E$ be an elliptic curve over $\mathbb{C}$.  Then there exists a lattice
   $\Lambda$ in $\mathbb{C}$ such that $E \cong E_{\Lambda}$, where $E_{\Lambda}$ is
   given by \eqref{eq:complex-eqn-for-ec}.
 \end{cor}
 \begin{proof}
   Suppose $E$ is given by the equation
   \begin{equation*}
     y^{2} = 4x^{3} + Ax + B.
   \end{equation*}
   Then, since $E$ is non-singular, its discriminant $A^{3} - 27B^{2}$ is non-zero.
   Now by Uniformisation there exists a lattice $\Lambda$ such that
   \begin{equation*}
     g_{2}(\Lambda) = A \quad \text{and} \quad g_{3}(\Lambda) = B,
   \end{equation*}
   so that the $j$-invariant $j(E_{\Lambda})$ of the elliptic curve $E_{\Lambda}$ is
   given by
   \begin{align*}
     j(E_{\Lambda})&= 1728 \cdot \frac{g_{2}^{3}(\Lambda)}{\Delta(E_{\Lambda})}\\
     &=1728 \cdot \frac{A^{3}}{A^{3} - 27B^{2}}\\
     &=j(E),
   \end{align*}
   so $E \cong E_{\Lambda}$ as required.
 \end{proof}

 When $E$ is an elliptic curve over $\mathbb{C}$, Proposition
 \ref{prop:complex-isogenies} together with Uniformisation imply that
 \begin{equation}
   \label{eq:End(E)-in-complex}
   End(E) \cong \{a \in \mathbb{C} : \alpha \Lambda \subset \Lambda \}
 \end{equation}
 for some unique lattice $\Lambda$ in $\mathbb{C}$.

%%% Local Variables:
%%% mode: latex
%%% TeX-master: "../main"
%%% End:
