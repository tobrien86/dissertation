
\section{Complex multiplication over $\mathbb{C}$}
\label{sec:compl-mult-over-C}

Thus far we have established the general structure of the endomorphism ring of an
elliptic curve, but we do not yet have a satisfactory way of determining whether a
given elliptic curve has complex multiplication or not.
TODO: more intro material.

\subsection{A brief review of elliptic curves over $\mathbb{C}$}
\label{sec:review-complex-curves}

TODO: Reference A Course in Arithmetic.  Let $\Lambda$ be a lattice in $\mathbb{C}$.
Recall the Weierstrass $\wp$-function:
\begin{equation*}
  \wp(z,\Lambda)\footnote{We will usually supress the $\Lambda$ and simply write $\wp(z)$} = \frac{1}{z^{2}} + \sum_{\omega \in \Lambda - 0}
  \frac{1}{(z-\omega)^{2}} - \frac{1}{\omega^{2}}.
\end{equation*}
The $\wp$-function is doubly-periodic, and thus descends to a well-defined function
on the torus $\mathbb{C} / \Lambda$.  Recall also the Eisenstein series for $\Lambda$
of weight $2k$:
\begin{equation*}
  G_{2k}(\Lambda) = \sum_{\omega \in \Lambda - 0} \frac{1}{\omega^{2}}.
\end{equation*}
There is a relation of algebraic dependence between $\wp$ and its derivative, given
by:
\begin{equation*}
  \wp'(z)^{2} = 4\wp(z)^{3} - g_{2}\wp(z) - g_{3},
\end{equation*}
where $g_{2} = 60G_{4}(\Lambda)$ and $g_{3} = 140G_{6}(\Lambda)$.  So if
$E_{\Lambda}$ is the curve in $\mathbb{P}^{2}_{\mathbb{C}}$ defined by the equation
\begin{equation}
  \label{eq:complex-eqn-for-ec}
  y^{2} = 4x^{3} - g_{2}x - g_{3},
\end{equation}
then there is a holomorphic bijection of Riemann surfaces $\mathbb{C}/\Lambda
\rightarrow E_{\Lambda}$ given by
\begin{equation*}
  z + \Lambda \rightarrow
  \begin{cases}
    [\wp(z) : \wp'(z) : 1] & z \neq 0,\\
    [0 : 1 : 0] & z = 0.
  \end{cases}
\end{equation*}
The curve defined in \eqref{eq:complex-eqn-for-ec} is nonsingular provided the
discriminant
\begin{equation*}
  \Delta(E_{\Lambda}) = g_{2}^{3} - 27g_{3}^{2}
\end{equation*}
is non-zero.

\subsection{Endomorphisms}
\label{sec:endomorphisms}

One advantage of working over the complex numbers is that the endomorphism ring of an
elliptic curve can be interpreted in a simple way in terms of the lattice which
defines the curve.

Let $\Lambda_{1}$ and $\Lambda_{2}$ be lattices in $\mathbb{C}$. Suppose $\alpha \in
\mathbb{C}$ is such that $\alpha\Lambda_{1} \subset \Lambda_{2}$.  Then $\alpha$
defines a map $\phi_{\alpha} \colon \mathbb{C}/\lambda_{1} \rightarrow
\mathbb{C}/\Lambda_{2}$ given by
\begin{equation*}
  \phi_{\alpha}(z + \Lambda_{1}) = \alpha z + \Lambda_{2}.
\end{equation*}
This map is well defined; if $\omega \in \Lambda_{1}$, then
\begin{equation*}
  \alpha(z + \omega) = \alpha z + \alpha\omega \equiv \alpha z \pmod{\Lambda_{2}}.
\end{equation*}
The maps $\phi_{\alpha}$ are clearly holomorphic group homomorphisms.

\begin{prop}
  \label{prop:complex-isogenies}
  TODO: Write this out in own words.
\end{prop}
\begin{proof}
  
\end{proof}

\begin{cor}
  \label{cor:homothetic-lattices-give-isomorphic-curves}
  Let $\Lambda_{1}$ and $\Lambda_{2}$ be as above.  Then the curves $E_{\Lambda_{1}}$
  and $E_{\Lambda_{2}}$ are isomorphic if and only if $\Lambda_{1}$ is homothetic to $\Lambda_{2}$.
\end{cor}

From the general theory of elliptic curves we know that for any non-zero $d$ in an
arbitrary field $k$ there exists an elliptic curve $E$ with discriminant $d$.  When
$k$ is the field of complex numbers there is a much stronger result, known as the
\emph{uniformisation thorem}, which states:

\begin{thm}
  \label{thm:uniformisation-theorem}
  Let $A$ and $B$ be complex numbers which satisfy
  \begin{equation*}
    A^{3} - 27B^{2} \neq 0.
  \end{equation*}
  Then there exists a unique lattice $\Lambda \subset \mathbb{C}$ such that
  \begin{equation*}
    g_{2}(\Lambda) = A \text{\quad and \quad} g_{3}(\Lambda) = B.
  \end{equation*}
\end{thm}
\begin{proof}
  TODO: Reference Shimura.
\end{proof}

An obvious consequence of the uniformisation is the following corollary:

\begin{cor}
  \label{cor:uniformisation-corollary}
  Let $E$ be an elliptic curve over $\mathbb{C}$.
\end{cor}

%%% Local Variables: 
%%% mode: latex
%%% TeX-master: "../main"
%%% End: 
